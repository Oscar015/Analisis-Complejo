\documentclass[10pt]{book}


\usepackage{amsmath, amsfonts, amssymb, amsthm} %paquetes matematicos
\usepackage{mathtools}
\usepackage{cancel}

%paquetes idioma
\usepackage[T1]{fontenc}
\usepackage[utf8]{inputenc}
\usepackage[spanish]{babel}
\usepackage{csquotes}

\usepackage{graphicx} %añadir imagenes

\usepackage{listings} %codigo python resaltado
\usepackage{xcolor} %colores

%\usepackage[a4paper,left=3cm,right=2.5cm,top=2.5cm,bottom=2.5cm]{geometry} %margenes y geometria de la hoja

\usepackage{verbatim} %entorno comment
\usepackage{subfig} %subfiguras
\usepackage{float} % parametro H en figuras

\usepackage[hidelinks]{hyperref} %Urls

\usepackage[affil-it]{authblk}

\usepackage{tikz}
\usepackage{tkz-euclide}
\usetikzlibrary{positioning,decorations.markings}

\tikzset{
every picture/.append style={
  execute at begin picture={\deactivatequoting},
  execute at end picture={\activatequoting}
  }
}

%paquetes de tablas
\usepackage{tabularx}
\usepackage{multirow}

\newcolumntype{M}{>{$}c<{$}} %Tipo de columna matematica

%Lineas viudas y huerfanas
\clubpenalty=10000
\widowpenalty=10000

\pretolerance=10000
\tolerance=10000

%Datos portada

\author{-----}
\affil
{
	Universidad de Alicante
    \vskip -1cm
}
% \author{Óscar Riquelme Moya}
% \affil
% {
%     Departamento de física aplicada\\
%     Universidad de Alicante
%     \vskip -1cm
% }

%Fuentes y colores para el highlight
\DeclareFixedFont{\ttb}{T1}{txtt}{bx}{n}{9} %bold
\DeclareFixedFont{\ttm}{T1}{txtt}{m}{n}{9}  %normal

\definecolor{font}{rgb}{0.2,0.1,0.2}
\definecolor{background}{rgb}{0.95,0.95,1}
\definecolor{deepblue}{rgb}{0,0,0.5}
\definecolor{deepred}{rgb}{0.6,0,0}
\definecolor{deepgreen}{rgb}{0,0.5,0}
\definecolor{grey}{rgb}{0.3,0.3,0.3}
\definecolor{red}{rgb}{1,0,0}

%estilo para el codigo python
\lstdefinestyle{Python}{language=python,
		backgroundcolor=\color{background},
		basicstyle=\ttm\color{font},             
		keywordstyle=\ttb\color{deepblue},  
		stringstyle=\color{deepgreen},
		showspaces=false,
		showstringspaces=false,
		tabsize=4,
		numbers=left,
		otherkeywords={self},
		emph = {__init__, as},
		emphstyle=\ttb\color{deepred},
		numberstyle=\color{grey},
		numbersep=6pt,
		stepnumber=1,
		rulecolor=\color{black},
		frame=single,
		framexleftmargin=20pt}
		


%Cambiamos el nombre de las tablas de Cuadro a Tabla
\AtBeginDocument{%
  \renewcommand\tablename{Tabla}
}

\newcommand{\parcial}[2]{
\frac{\partial #1}{\partial #2}
}


\newcommand{\parcialn}[3]{
\frac{\partial ^#3 #1}{\partial #2^#3}
}



%comando para hacer figuras rapidamente
% \fig[escala]{raiz/directorio../archivo}{caption}{label}
\newcommand{\fig}[4][1]{
\begin{figure}[H]
\begin{center}
\includegraphics[scale=#1]{#2}
\caption{#3}\label{#4}
\end{center}
\end{figure}
}

%Comandos para referenciar rápicdamente figuras tablas y ecuaciones
\newcommand{\rfig}[1]{
(figura \ref{#1})
}

\newcommand{\rtabla}[1]{
(tabla \ref{#1})
}
\newcommand{\req}[1]{
(ecuación \ref{#1})
}
\newcommand{\abs}[1]{\left| #1\right|}

%Entorno paar escribir codigo python
\lstnewenvironment{Python}{
\lstset{style=Python}}{}

\newcommand{\Pythonline}[1]{
\lstinline[style=Python]|#1|}



\setlength{\parskip}{0.5em}
\usepackage{titlesec}
\titleformat{\chapter}[block]
  {\normalfont\Huge\bfseries}{Bloque \thechapter \\ }{1em}{\huge}

%Teoremas matemáticos
\newtheorem{defi}{Definición}[chapter]
\newtheorem{theorem}{Teorema}[chapter]
\newtheorem{prop}{Proposición}[chapter]
\newtheorem*{dem}{Demostración}
\newtheorem{col}{Colorario}[chapter]
\newtheorem{lema}{Lema}[chapter]
\newtheorem{ejemplo}{Ejemplo}[chapter]
\captionsetup[subfigure]{labelformat=empty}

\newcommand{\deriv}{\displaystyle \lim_{h\to 0} \frac{f(z_0+h)-f(z_0)}{h}}
\newcommand{\R}{\mathbb{R}}
\newcommand{\C}{\mathbb{C}}
\newcommand{\N}{\mathbb{N}}
\newcommand{\Z}{\mathbb{Z}}

\newcommand{\Log}{\emph{Log}}
\newcommand{\Arg}{\emph{Arg}}
\newcommand{\Res}{\emph{Res}}

\newcommand{\f}{f: U\subset \C \longrightarrow \C}

\title{Análisis de variable compleja}

\begin{document}
\maketitle
\tableofcontents
\chapter*{Preámbulo}
\addcontentsline{toc}{chapter}{Preámbulo}
Estos apuntes están destinados a complementar los apuntes tomados por
los estudiantes de la asignatura homónima perteneciente al tercer año del
Grado en Física de la Universidad de Alicante.
Es un texto realizado fundamentalmente a partir de las notas tomadas
durante las lecciones impartidas por el profesor Juan Matías Sepulcre Martínez, del departamento
de Análisis Matemático, durante el curso académico 2021-22022, destinado exclusivamente a estudiantes y sin ánimo de lucro.
No está exento de erratas. La edición de estos apuntes se remite a la fecha de compilación que aparece en la portada. El último tema es el que menos horas hemos echado por falta de las mismas. 

Para obtener una copia del código fuente o para comunicar posibles erratas, o colaborar de cualquier forma para mejorar estos apuntes se ruega contactar a: {\color{blue}\href{mailto:orm13@alu.ua.es?subject=Apuntes análisis complejo}{orm13@alu.ua.es}}


\chapter{El cuerpo de los números complejos.}
\section{Los números complejos}

\begin{defi}
Sean $a,b \in\R$ entonces $z = a + ib \in \C$.

Definimos la parte real de $z$ como $\Re(z) = Re(z) = a$ y la parte imaginaria de $z$ como $\Im(z) = Im(z) = b$
\end{defi}


 $\alpha = \arctan (\frac{b}{a}) \longrightarrow \text{ Argumento principal: } (-\pi,\pi)$
 
 \textbf{Forma trigonométrica}: $z = a+bi = |z|(\cos \alpha + i\sin \alpha)$
 
 \textbf{Forma exponencial}: $z = a+bi = |z|e^{i\alpha}$
$$ 
	Arg(z)= \left\{ \begin{array}{lcc}
	\frac{\pi}{2} & si & a=0,b>0 \\
	\frac{-\pi}{2} & si & a=0,b<0\\
	\arctan (\frac{b}{a}) & si & a>0,b>0\\
	-\arctan (\frac{-b}{a}) & si & a>0, b<0\\
	-\arctan (\frac{b}{-a}) + \pi & si &  a<0,b>0\\
	\arctan (\frac{b}{a}) - \pi & si & a<0,b<0\\
	0 & si & a>0,b=0\\
	\pi & si & a<0, b=0
	\end{array}
\right.  
$$


\textbf{Formula de Moivre}: $z^n = (re^{i\alpha})^n = r^n e^{in\alpha} = r^n(\cos(n\alpha) + i \sin(n\alpha))$
\newpage
\section{Analiticidad}
$$\text{Sea }\begin{array}{ccc}
f : \mathbb{C} &\longrightarrow& \mathbb{C}\\
z &\longmapsto& f(z)\end{array}$$

\begin{defi}[Límite]
	Definimos el límite de $f(z)$ cuando $z$ tiende a $z_0$ como:
$$ \lim_{z \to z_0} f(z)= w_0 \Leftrightarrow \forall \varepsilon >0 \exists \delta / \text{ si  } |z-z_0| < \delta \Rightarrow |f(z) -w_0|<\varepsilon$$
\end{defi}

\begin{defi}[Continuidad]
f es continua en $z_0 \Leftrightarrow \forall \varepsilon>0$ $\exists \delta /$ si $|z-z_0| < \delta \Rightarrow |f(z)-f(z_0)| < \varepsilon.$
\end{defi}

\begin{defi}[Derivavilidad]
f es derivable en $z_0 \Leftrightarrow$ 
$$\exists f'(z_0) = \lim_{h \to 0} \frac{f(z_0+h)-f(z_0)}{h}$$
\end{defi}


\begin{defi}[Analiticidad]
Sea $f: U \subset \C \longrightarrow \C$ una función compleja definida en un abierto $ U \subset \mathbb{C}$, diremos que $f$ es analítica en $U$ cuando exista un entorno de $U$ donde $f$ es derivable en todo punto.
\end{defi}

\begin{defi}
Se dice que una función $f$ es entera si es analítica en todo $\C$.
\end{defi}

\begin{theorem}[Ecuaciones de Cauchy-Riemman]
Sea $f: U \subset \C \longrightarrow \C$ , $f(z) = f(x+iy) = u(x,y) + iv(x,y)$ analítica n el abierto $U$, entonces se satisface:
$\parcial{u}{x}(x_0,y_0) = \parcial{v}{y}(x_0,y_0)$ y $\parcial{u}{y}(x_0,y_0) = -\parcial{v}{x}(x_0,y_0)$ $ \forall(x_0,x_y) \in U$
\end{theorem}

\begin{dem}
Sea $z_0 = x_0 + iy_0 \in U \Rightarrow$ $\exists f'(z_0) = \deriv$

Tomamos $h \in \R$

\begin{multline*}
\lim_{h\to 0} \frac{f(x_0+h+y_0i)-f(x_0+y_oi)}{h} =\\
 \lim_{h\to 0}\frac{(u(x_0+h,y_0)-u(x_0,y_o)) + (v(x_0+h,y_0)-v(x_0,y_o))i}{h} =\\
  \parcial{u}{x}(x_0,y_0) + \parcial{v}{x}(x_0,y_0)
  \end{multline*}


Sea $k \in\R$, tomamos $h=ik$:

\begin{multline*}
\lim_{h=ik\to 0} \frac{f(x_0+(y_0+y)i)-f(x_0+y_oi)}{ik} =\\
 \lim_{h=ik\to 0} \frac{(u(x_0,(y_0+y))-u(x_0,y_oi))+(v(x_0,(y_0+y))-v(x_0,y_oi)i)}{ik} =\\
  -\parcial{u}{x}(x_0,y_0)i + \parcial{v}{x}(x_0,y_0)i
  \end{multline*}

$$
\Rightarrow \parcial{u}{x}(x_0,y_0) + \parcial{v}{y}(x_0,y_0)i = \parcial{v}{y}(x_0,y_0) - \parcial{u}{y}(x_0,y_0)i
$$$$
\Rightarrow \parcial{u}{x} = \parcial{v}{y} \hspace{0.5cm} \parcial{u}{y} = -\parcial{v}{x}\ \qed
$$
\end{dem}


\begin{theorem}
Sea $\f$, $f(z) = f(x+yi) = u(x,y)+ v(x,y)i$ definida en $U$ abierto tal que existen las parciales primeras de $u$ y $v$ en un entorno de $(x_0,y_0) \in V$ y son continuas, entonces si se verifican las ecuaciones de Cauchy-Riemann en $(x_0,y_0)$ entonces $f$ es derivable en $z_0 = x_0 + y_0i$.
\end{theorem}

\begin{prop}[Cauchy-Riemann en coordenadas polares]
\begin{multline*}
x = r\cos\theta,\ y = r\sin\theta\\
\parcial{u}{r}(r_0,\theta_0) = \frac{1}{r_0}\parcial{v}{\theta}(r_0,\theta_0),\ \frac{1}{r_0}\parcial{u}{\theta}(r_0,\theta_0) = -\parcial{v}{r}(r_0,\theta_0)\\
\Rightarrow f'(z) = \parcial{u}{x}(x_0,y_0) + \parcial{v}{y}(x_0,y_0)i = (\parcial{u}{r}(r_0,\theta_0) + i\parcial{v}{\theta}(r_0,\theta_0))(\cos\theta -i\sin\theta)
\end{multline*}
\end{prop}

\section{Algunas funciones elementales}
\subsection{Función exponencial}
Sea $x,y \in\R \text{ y sea } z = x+iy$
$\Rightarrow e^{x} = \sum_{n=0}^\infty \frac{x^n}{n!}$
\begin{multline*}
e^{iy} = \sum{n=0}^\infty \frac{(iy)^n}{n!} = 1 + iy - \frac{y^2}{2}-\frac{y^3}{3!}i + \cdots\\
= (1-\frac{y^2}{2} + \frac{y^4}{4!}+\cdots) +i(y-\frac{y^3}{3!}+\cdots) = \cos y + i \sin y\\
e^x + e^{iy} = e^{x+iy} \Rightarrow e^{z} = e^x(\cos y + i \sin y)
\end{multline*}
\subsubsection*{Propiedades}
\begin{enumerate}
\item $e^z \neq 0 \forall z in \C$
\item $|e^{z}| = e^{Re z}$
\item $\overline{e^z} = e^{\overline z}$
\item $(e^z)^n = e^{zn}$
\item $e^{z_1+z_2} = e^{z_1}z^{z_2}$
\item $ e^z = 1 \Leftrightarrow e^x e^{iy}=1 \Leftrightarrow x = 0 \wedge y = 2\pi k \Leftrightarrow z = 2\pi ki, k\in \Z$
\item $\displaystyle \nexists \lim_{z \to 0} e^z$
\end{enumerate}

\subsection{Función Logarítmica}
$e^w = z \Leftrightarrow \log z = w$ con $z \neq 0 \in \C$.

Si $z=1 \Rightarrow e^w = 1 \Leftrightarrow w = 2\pi k i, k \in \Z$.
 
\begin{defi}[Logaritmo principal]
Sea $z\neq 0$, entonces el logaritmo principal de $z$, denotado por $\Log$, viene definido por $\Log z = \ln(|z|) + i \Arg(z)$ donde $\Arg(z)\in (-\pi,\pi]$. Denotamos por $\log(z)$ al conjunto de elementos $w \in \C$ de forma que $e^w = z$. 
\end{defi}


\begin{defi}
Sea $\theta_0 \in\R$ Definimos $\log_{\theta_0} z = \ln |z| + i\theta$ con $\theta \in (\theta_0,\theta_0+2\pi]$.

Podemos ver como $\Log z = \log_{-\pi} z$.
\end{defi}

\subsubsection*{Propiedades}
\begin{enumerate}
\item $\log(1) = \{2\pi ki, k\in\Z \}$
\item $\log(-1) = \{(2k+1)\pi i, k\in\Z \}$
\item $\log(z_1z_2) = log(z_1)+\log(z_2)$ pero no siempre se cumple que 

$\Log(z_1z_2) = \Log(z_1) + \Log(Z_2)$.
\item $\Log z$ es una función continua en $\C \backslash (-\infty,0]$
\item  $\Log z$ es una función analítica en $\C \backslash (-\infty,0]$
$$ \frac{d}{dz}Log z = \frac{1}{z} \ \forall z \in \C \backslash (-\infty,0]$$
\item $e^{log_{\theta_0} z} = z \ \forall z \in \C \backslash\{0\}$
\item $log_{\theta_0} e^z=z \ \forall z \in\{x+iy / \theta_0 \leq y \leq \theta_0 + 2\pi\}$
\end{enumerate}

\subsection{Funciones potencia}
\begin{defi}
Sea $z\neq 0$ y $\alpha \in\C$ tomaremos por definición $z^\alpha$, llamada potencia de exponente arbitrario $\alpha$, como el conjunto de todos los valores dados por:
$$z^\alpha = e^{\alpha\log z}$$
\textbf{Nota:} Si $alpha \in\R$ entonces $z^\alpha = |z|^\alpha e^{i\alpha\arg z}$.
\end{defi}


\begin{defi}
Sea $a \in\C\backslash\{0\}$, tomaremos por definición $a^z$, llamada la función exponencial general, como el conjunto de todos los valores dados por:
$$a^z = e^{z\log a}$$
\textbf{Nota:} Para obtener ramas uniformes de la función exponencial basta con fijar uno de los valores del logaritmo. Cuando se toma $\Log z$ se llama rama principal.
\end{defi}

\subsection{Funciones trigonométricas e hiperbólicas}
\begin{align*}
e^{it} = \cos t + i\sin t &  \hspace{1.5cm} \cos t = \frac{e^{it}+e^{-it}}{2} \\
&\ \Longrightarrow&  \\
e^{-it} = \cos t - i\sin t & \hspace{1.5cm}  \sin t = \frac{e^{it}-e^{-it}}{2i}
\end{align*}

\begin{align*}
\tan z &= \frac{\sin z}{\cos z} = \frac{e^{it}+e^{-it}}{i(e^{it}-e^{-it})}\\
\sinh z &= \frac{e^{z}-e^{-z}}{2} \  \cosh z = \frac{e^{z}+e^{-z}}{2} \\
\tanh z &= \frac{\sinh z}{\cosh z} = \frac{e^{it}+e^{-it}}{e^{it}-e^{-it}}
\end{align*}


\begin{enumerate}
\item $\overline{\sin z} = \sin \overline{z}$
\item $\overline{\cos z} = \cos \overline{z}$
\item $\sin z = 0 \Leftrightarrow z = \pi k, k \in\Z$
\item $\cos z = 0 \Leftrightarrow z = \frac{\pi}{2}+2\pi k, k\in\Z$
\item $\sinh z = 0 \Leftrightarrow z = \pi ki, k \in\Z$
\item $\cosh z = 0 \Leftrightarrow z = \frac{\pi}{2}(2k+1)\, k\in\Z$
\item $\sinh z = -\sin(iz)$
\item $\cosh z = \cos(iz)$

Tomando $z = x+iy$:
\item $\sin z = \sin x\cosh y + i\cos x sinh y$
\item $\cos z = \cos x\cosh y -i\sin x \sinh y$
\item $|\cos z|\geq |\cos x|$
\item $|\sin z|\geq |\sin x|$
\end{enumerate}


\begin{align*}
e^z &= \sum_{j=0}^{\infty} \frac{z^j}{j!}\\
\sin z &= \sum_{j=0}^{\infty} \frac{(-1)^j z^{2j+1}}{(2j+1)!} &\sinh z &= \sum_{j=0}^{\infty} \frac{z^{2j+1}}{(2j+1)!} \\
\cos z &= \sum_{j=0}^{\infty} \frac{(-1)^j z^{2j}}{(2j)!}   &\cosh z &= \sum_{j=0}^{\infty} \frac{z^{2j}}{(2j)!}  
\end{align*}

\begin{defi}
Se dice que $u: \R^2 \longrightarrow \R$ es armónica en un dominio $D$ si tiene derivadas parciales segundas continuas en todo punto de $D$ y satisface la ecuación de Laplace $\displaystyle \parcialn{u}{x}{2} +  \parcialn{u}{y}{2} =0$ en todo punto de $D$.
\end{defi}

\begin{defi}
Se dice que $v(x,y)$ es una conjugada armónica de $u(x,y)$ en $D\subset\R^2$ si ambas son armónicas y verifican las ecuaciones de Cauchy-Riemann en $D$, es decir, existe una función compleja $f(x+iy) = u(x,y)+iv(x,y)$ analítica en $D$.
\end{defi}

\begin{prop}
Sea $u(x,y)$ una función armónica en $D$ y consideremos $U$ una región rectangular contenida en $D$ entonces existe una conjugada armónica de $u(x,y)$ en  $U$.
\end{prop}

\begin{dem}
Sea $P(x,y)dx + Q(x,y)dy =0$ donde $P(x,y)= -\parcial{u}{y}$ y $Q(x,y)= \parcial{u}{x}$. Como $u(x,y)$ es armónca entonces $\parcial{P}{y} = \parcial{Q}{x}$ que es una ecuación diferencial exacta y existe $v: U \longrightarrow \R$ tal que 
$$dv = P(x,y)dx + Q(x,y)dy \Leftrightarrow \parcial{v}{x} = P = -\parcial{u}{y},\ \parcial{v}{y} = Q = \parcial{u}{x}$$
con lo que $v(x,y)$ es conjugada armónica de $u(x,y)$. $\qed$
\end{dem}
\chapter{Integración en $\C$}
\section{Preliminares topológicos}

\begin{defi}[Entorno perforado]
Sea $z_0 \in\C$ llamamos entorno perforado al conjunto abierto definido or $\{z\in\C / 0 < |z-z_0|<\varepsilon\}$ para algún $\varepsilon>0$.
\end{defi}

\begin{defi}[Conjunto conexo]
LLamamos conjunto conexo al conjunto abierto (cerrado) del plano  que no puede ser escrito como unión disjunta de dos conjuntos no vacíos abiertos (cerrados).
\end{defi}

\begin{defi}[Conjunto poligonalmente conexo]
Decimos que $P$ es un conjunto poligonalmente conexo si cada par de puntos  de $P$ pueden ser unidos mediante un polígono contenido en $P$.
\end{defi}

\begin{defi}[Conjunto estrellado]
Decimos que $E$ es un conjunto estrellado si $\exists a\in E / [a,z]\subset E\ \forall z \in E$.
\end{defi}

\begin{defi}[Conjunto convexo]
Decimos que $C$ es un conjunto convexo si $[z,w]\subset C\ \forall z,w \in C$
\end{defi}

\begin{defi}[Conjunto simplemente convexo]
Decimos que $S$ es un conjunto simplemente conexo cuando cualquier camino cerrado en $S$ puede deformarse de forma continua hasta convertirse en un punto sin salirse de $S$.
\end{defi}

\section{Integración sobre caminos}

\begin{defi}
LLamamos curva a una aplicación continua $\gamma: [a,b] \longrightarrow \C$ con $a<b$ tal que a un número real $t$ le corresponde un número complejo.
$$\gamma(t) = x(t) + iy(t)$$
La traza o trayectoria de la curva $\gamma([a,b]) = \{\gamma(t), t\in[a,b]\}$ será representado por $\gamma^*$.

Cuando $\gamma(a) = \gamma(b)$ se dice que la curva es cerrada.
\end{defi}

\begin{defi}
Sea $\gamma:[a,b]\longrightarrow \C$ una curva, si $\gamma(t) \neq \gamma(s) $ $\forall t\neq s$ excepto, a lo sumo, en los extremos, la curva será llamada un arco simple o arco simple de Jordan.
\end{defi}

\begin{defi}
Sea $\gamma:[a,b]\longrightarrow \C$ una curva, se dice que es diferenciable con continuidad, $\mathcal{C}^1([a,b])$, cuando $\gamma$ es diferenciable y presenta derivada continua en $[a,b]$. Además, $\gamma$ es diferenciable con continuidad a trozos cuando $[a,b]$ se puede descomponer en un número finito de subintervalos sobre los que $\gamma$ es diferenciable con continuidad.
\end{defi}

\begin{defi}
Llamaremos camino a una curva diferenciable con continuidad a trozos.

\end{defi}

\begin{defi}
Sea  $\gamma:[a,b]\longrightarrow \C$ un camino y $f(z)$ una función continua en $\gamma^*$, definimos la integral compleja de $f(z)$ a lo largo de $\gamma$ por:
$$
\int_{\gamma} f(z) dz = \int_a^b f(\gamma(t)) \gamma'(t)dt
$$
\end{defi}

\subsection*{Parametrizaciones comunes}
Intervalo $[z_1,z_2]$: $\gamma(t) = (1-t)z_1 +t \, z_2$ con $t\in [0,1]$.

Circunferencia de centro $z_0$ y radio $r$ $C(z_0,r)$: $\gamma(t) = z_0 + re^{it}$ con $t\in[0,2\pi]$ 

\begin{prop}
Sea $\gamma:[a,b]\longrightarrow \C$ un camino y $f(z), g(z)$ funciones continuas en $\gamma^*$ entonces se cumplen las siguientes propiedades:
\begin{enumerate}
\item La integral sobre un camino es invariante de parametrizaciones.
\item $\int_\gamma (f(z)+g(z)) dz = \int_\gamma f(z)dz + \int_\gamma g(z) dz$
\item $\int_\gamma c f(z) dz = c \int_\gamma f(z) dz, c\in\C$
\item $\int_{-\gamma} f(z) dz = -\int_\gamma f(z) dz$ donde $-\gamma$ es el camino opuesto definido por $(-\gamma)(t) = \gamma(b+a-t) \ \forall t \in [a,b]$
\item Si $\beta$ es otro camino tal que $\gamma + \beta$ está definido y $f(z)$ es tambien con tinua en $(\gamma+\beta)^*$ entonces:
$$
\int_{\gamma+\beta}f(z)dz = \int_\gamma f(z)dz + \int_\beta f(z)dz
$$
\end{enumerate}
\end{prop}


\begin{prop}
Sea $\gamma:[a,b]\longrightarrow \C$ un camino y $f(z)$ una función continua en $\gamma^*$, entonces:
$$\abs{\int_\gamma f(z)dz }\leq M_f(\gamma)L(\gamma)$$
Con $M_f(\gamma) = \max \{|f(z)|,z \in \gamma^* \}$ y $L(\gamma)$ la longitud de $\gamma$.
\end{prop}

\begin{dem}
$$\abs{\int_\gamma f(z)dz } = \abs{\int_a^b f(\gamma(t))\gamma'(t)dt}\leq \int_a^b \abs{f(\gamma(t))}\abs{\gamma'(t)}dt\leq M_f(\gamma)\int_a^b\abs{\gamma'(t)}dt =  M_f(\gamma)L(\gamma)$$
\end{dem}

\begin{defi}
Sea $\f$ una función definida en un abierto $U$, diremos que $F: U \longrightarrow\C$ es una primitiva de $f$ en $U$ cuando $F(z)$ es analítica en $U$ y $F'(z) = f(z) \ \forall z \in\C$.
\end{defi}

\begin{theorem}[Extensión del $2^\circ$ teorema fundamental del cálculo]
Supongamos que $f(z)$ es una función continua en un conjunto abierto $U$ y que $F(z)$ es una primitiva de $f$ en $U$. Si $\gamma:[a,b]\longrightarrow \C$ es un camino en $U$ entonces:
$$\int_\gamma f(z)dz = F(z)\bigg|_{\gamma(a}^{\gamma(b)}$$
En particular, bajo las hipótesis anteriores si $\gamma$ es un camino cerrado: $\int_\gamma f(z)dz = 0$.
\end{theorem}

\begin{dem}
Sea $G(t) = F(\gamma(t)$ con $t\in [a,b]$ entonces $G(t)$ es continua en $[a,b]$ y $G'(t) = F'(\gamma(t))\gamma'(t)$ con lo que
$$\int_\gamma f(z)dz = \int_a^b f(\gamma(t)\gamma'(t)dt = \int_a^b G'(t)dt = G(t)\bigg|_a^b = F(\gamma(t)\bigg |_a^b $$ $\qed$
\end{dem}

\begin{col}[Funciones constantes]
Supongamos que $f(z)$ es analítica en un abierto y conexo $U$ y además $F'(z)=0 \ \forall z \in U$ entonces $f(z)$ es constante en $U$.
\end{col}

\begin{dem}
Sean $z_1$ y $z_2$ en $U$. Dado que $U$ es conexo, existe un camino poligonal $\gamma:[a,b] \longrightarrow \C$ que los une tal que $\gamma(a) = z_1$ y $\gamma(b) = z_2$. Por el teorema anterior tenemos que
$$0 = \int_{\gamma} f'(z)dz = f(z_2)-f(z_1) \Rightarrow f(z_2)=f(z_1) \Rightarrow f \text{ es constante} \qed$$
\end{dem}


\begin{theorem}[Independencia de caminos]
Supongamos que $f(z)$ es una función continua en el conjunto abierto y conexo $U$. Las siguientes propiedades son equivalentes:
\begin{enumerate}
\item $\displaystyle\int_{\gamma}f(z)dz$ es independiente de la trayectoria de $\gamma$.
\item  $\displaystyle\int_{\gamma}f(z)dz = 0$ $\forall \gamma$ cerrado.
\item $f(z)$ admite una primitiva en $U$.
\end{enumerate}
\end{theorem}


\begin{dem}
$\underline{1 \Leftrightarrow 2}$

Sea $\gamma_1$ un camino en $U$ que une $z_1$ y $z_2$, sea $\gamma_2$ otro camino en $U$ que une $z_1$ y $z_2$.

Consideremos $\gamma = \gamma_1 -\gamma_2$ un camino cerrado, entonces:
$$\int_\gamma f(z)dz = \int_{\gamma_1 -\gamma_2} f(z)dz = \int_{\gamma_1} f(z)dz - \int_{\gamma_2} f(z)dz$$
Si $\int_\gamma f(z)dz$ entonces la integral sobre $\gamma_1$ es igual a la integral sobre $\gamma_2$ con lo que es independiente del camino.

De la misma forma, si la integral es independiente del camino entonces $\displaystyle\int_{\gamma_1} f(z)dz - \int_{\gamma_2} f(z)dz = 0$.

$\underline{3 \Rightarrow 1}$

Por el 2º Teorema fundamental del Cálculo.

$\underline{1 \Rightarrow 3}$

Fijemos $z_0 \in \C$ y consideremos $\gamma_z$ un camino en $U$ que une $z_0$ con $z$, siendo $z \in U$.

Sea $F(z) = \int_{\gamma_z}f(s)ds$, por hipótesis $F(z)$ no depende del camino. Veamos que $F$ es analítica y $F'=f \ \forall z \in U$.

Dado que $U$ es abierto, podemos tomar $w\in U$ suficientemente próximo a $z$ con $w\neq z$. Entonces si consideramos $\gamma$ al camino que une $z_0$ consigo mismo pasando por $\gamma_z$, $[z,w]$ y $-\gamma_w$. Entonces
$$
\int_\gamma f(z)dz = \int_{\gamma_z} f(z)dz + \int_{[z,w]} f(z)dz + \int_{\gamma_z} f(z)dz = 0
$$ 

Por tanto $F(w)-F(z) = \int_{\gamma_w} f(z)dz - \int_{\gamma_z} f(z)dz = \int_{[z,w]} f(z)dz$ y al ser $\int_[z,w] 1ds = z-w$ se tiene que:
\begin{multline*}
f(z) = \frac{f(z)}{w-z}\int_{[z,w]}ds= \frac{1}{w-z}\int_{[z,w]}f(z)ds \Rightarrow \frac{F(w)-F(z)}{w-z}-f(z) =\\
\frac{\int_{[z,w]}f(z)ds}{w-z}-\frac{1}{w-z}\int_{[z,w]}f(z)ds = \frac{\int_{[z,w]}(f(s)-f(z))ds}{w-z}
\end{multline*}
El módulo tiende a $0$ por ser $f$ continua cuando $w \to z$, entonces 
$$F'(z) = \lim_{w\to z}\frac{F(w)-F(z)}{w-z} = f(z)$$
$\qed$

\end{dem}

\section{Teoremas de Cauchy para determinadas regiones}


\begin{theorem}[Teorema de Cauchy para triángulos]
Sea $f(z)$ una función analítica en un abierto $U$. Para cualquier triangulo ($\triangle = int(\triangle \cup Fr(\triangle)$) contenido en $U$ se tiene que:
$$
\int_{\Gamma} f(z)dz = 0\text{ donde }\Gamma =Fr(\triangle)
$$
\end{theorem}

\begin{dem}
Sea $\Gamma_0$ el camino de la frontera del triangulo, podemos subdividir este en 4, teniendo 4 subtriángulos con fronteras $\Gamma_{0,1},\ldots,\Gamma_{0,4}$, y este proceso lo podemos repetir las veces que queramos. Podemos verlo gráficamente así:

\begin{figure}[H]
\begin{center}
\begin{tikzpicture}[scale=0.1]

% puntos
\tkzDefPoints{0/0/A, 50/0/B, 32/24/C, 16/12/D, 25/0/E, 41/12/F, 20.5/6/G, 8/6/H, 12.5/0/I, 14.25/6/J, 16.5/3/K, 10.25/3/L}
\tkzDrawPoints(A, B, C)	

% vértices del triángulo
\draw (A)--(B);
\draw (B)--(C);
\draw (C)--(A);
\draw (C) node[above] {c};
\draw (B) node[right] {b};
\draw (A) node[left] {a};

% flechas de los lados exteriores
\draw[->] (C)--(24.47, 18.35);
\draw[->] (B)--(35.79, 18.95);
\draw[->] (B)--(45.13, 6.49);
\draw[->] (A)--(40, 0);
\draw[->] (C)--(12.31, 9.23);
\draw[->] (C)--(4.35, 3.26);
\draw[->] (A)--(6.26, 0);
\draw[->] (A)--(20, 0);

% Triángulo interior (1)
\draw[->, dashed] (D)--(22.25, 12);
\draw[dashed] (22.25, 12)--(34.75, 12);
\draw[<-, dashed] (34.75, 12)--(F);

\draw[->, dashed] (E)--(29, 3);
\draw[dashed] (29, 3)--(37, 9);
\draw[<-, dashed] (37, 9)--(F);

\draw[->, dashed] (D)--(18.25, 9);
\draw[dashed] (18.25, 9)--(22.75, 3);
\draw[<-, dashed] (22.75, 3)--(E);


% Triángulo interior (2)
\draw[->, dashed] (H)--(11.13, 6);
\draw[dashed] (11.13, 6)--(17.38, 6);
\draw[<-, dashed] (17.38, 6)--(G);

\draw[->, dashed] (G)--(18.5, 4.5);
\draw[dashed] (18.5, 4.5)--(14.5, 1.5);
\draw[<-, dashed] (14.5, 1.5)--(I);

\draw[->, dashed] (I)--(11.38, 1.5);
\draw[dashed] (11.38, 1.5)--(9.13, 4.5);
\draw[<-, dashed] (9.13, 4.5)--(H);


% Triángulo interno (3)
\draw[dashed, color=red] (J)--(K);
\draw[dashed, color=red] (K)--(L);
\draw[dashed, color=red] (L)--(J);

\end{tikzpicture}
\end{center}
\end{figure}
	
\begin{multline*}
\int_\Gamma f(z)dz = \sum_{j=1}^4\int_{\Gamma_{0,j}} f(z) dz \Rightarrow \abs{\int_\Gamma f(z)dz)}\leq 4\abs{\int_{\Gamma_1}f(z)dz} \leq\\
 4^2\abs{\int_{\Gamma_2}f(z)dz}\leq 4^n \abs{\int_{\Gamma_n} f(z)dz}
\end{multline*}
$$\Gamma_k = \max_{j=1,\ldots, n} \left\{\abs{\int_{\Gamma_{k,j}}f(z)dz}\right\}$$
Sea $\bar{V_n} = V_n \cup \Gamma_n$ entonces tenemos que $\bar{V}_n\subset \bar{V}_{n-1}\subset \cdots \subset \bar{V}_1$. Y por el teorema de Cantor debe existir un $z_0$ tal que 
$$\bigcap_{n=1}^\infty V_n = \{z_0\}$$

$f$ es analítica en $U$ entonces $$f'(z_0) = \lim_{z\to z_0}= \frac{f(z)-f(z_0)}{z-z_0}\Leftrightarrow $$ $$\forall \varepsilon>0\ \exists \delta>0 / \text{ si } |z-z_0|<\delta\Rightarrow \abs{\frac{f(z)-f(z_0)}{z-z_0}-f'(z_0)} <\varepsilon $$

\begin{multline*}\forall z\in U, \ g(z) := f(z) -f(z_0) -(z-z_0)f'(z_0),\ \forall \varepsilon>0 \Rightarrow\\
 |g(z)|\leq \varepsilon|z-z_0 \text{ si } |z-z_0|< \delta|
 \end{multline*}
 \begin{align*}
 &Dado \delta>0,\ \exists n\in\N \ /\ \bar{V}_n \subset D(z_0,\delta)\\
 & \Rightarrow \int_{\Gamma_n}zdz = \frac{z_2^2 -z_1^2}{2}+\frac{z_3^2 -z_2^2}{2}+\frac{z_1^2 -z_3^2}{2} = 0, \ \int_{\Gamma_n} Cdz = 0,\ \forall C \in \C\\
 & \Rightarrow \abs{\int_{\Gamma_n}f(z)dz} = \abs{\int_{\Gamma_n}g(z)dz}  \leq M_g(\Gamma_n)L(\Gamma_n)\\
 & \Rightarrow \abs{\int_{\Gamma_n}f(z)dz} \leq M_{\varepsilon|z-z_0|}(\Gamma_n)L(\Gamma_n) \leq \varepsilon L(\Gamma_n)^2 \leq \varepsilon ( \frac{1}{2^n}L(\Gamma)^2\\
 & \Rightarrow \abs{\int_{\Gamma}f(z)dz} \leq 4^n \varepsilon (\frac{1}{2^n}L(\Gamma)^2 = \varepsilon L(\gamma)^2 \\
 & \Rightarrow \abs{\int_{\Gamma}f(z)dz} = 0\text{ porque podemos hacer }\varepsilon \text{ tan pequeño como queramos. }\qed
 \end{align*}
\end{dem}


\begin{theorem}[Teorema de Cauchy para regiones estrelladas]
Sea $f(z)$ una función analítica en un conjunto estrellado y abierto $U$. Para cualquier $\gamma$ camino cerrado en $U$ se tiene que:
$$
\int_{\gamma} f(z)dz = 0
$$
\end{theorem}

\begin{dem}
Al ser una región estrellada sabemos que $\exists z_0\in U$ tal que $[z_0,z]\in U\ \forall z \in U$.

Sea $F(z) = \int_{[z_0,z]}f(w)dw$ y veamos que $F'(z_1) = f(z_1)\ \forall z_1 \in U$.

Sea $z \in U,\ \neq z \neq z_0,\ z_1 \neq z \Rightarrow$
$$
\frac{F(z)-F(z_1)}{z-z_1} = \frac{\int_{[z_0,z]}f(w)dw- \int_{[z_0,z_1]}f(w)dw}{z-z_1} = \frac{\int_{[z_0,z]}f(w)dw+ \int_{[z_1,z_0]}f(w)dw}{z-z_1}
$$
Por el Teorema de caucy para triángulos, tomando $\Gamma = [z_0,z]\cup [z,z_1]\cup[z_1,z_0]$ vemos que $\int_\Gamma f(w)dw = 0$. Con lo que 
$$\frac{F(z)-F(z_1)}{z-z_1} = \frac{\int_{[z_1,z]}f(w)dw}{z-z_1}$$

$$
\abs{ frac{\int_{[z_1,z]}f(w)dw}{z-z_1}}\leq M_f(\gamma) \frac{|z-z_1|}{z-z_1} \xrightarrow[z\to z_0]{} f(z)
$$
$$
\Rightarrow \abs{\frac{F(z)-F(z_1)}{z-z_1}-f(z_1)}\xrightarrow[z\to z_0]{} 0 \Rightarrow F'(z_1) = f(z_1)
$$
Podemos concluir que $\int_\gamma f(z)dz=0$ para todo $\gamma$ camido cerrado. $\qed$
\end{dem}


\begin{theorem}[Teorema de Cauchy extendido para triángulos]
Sea $f(z)$ una función continua en un abierto $U$ y analítica en $U\backslash \{z_0\}$ para algún $z_0\in U$, para cualquier $\triangle$ contenido en $U$ se tiene que:
$$
\int_{\Gamma} f(z)dz = 0\text{ donde }\Gamma =Fr(\triangle)
$$
\end{theorem}
\begin{dem}
Vamos a dividir la demostración en casos:

\begin{figure}[H]
  \centering
  \subfloat[CASO 0]{
  	\begin{tikzpicture}[scale=0.1]
	% puntos
	\tkzDefPoints{0/0/A, 50/0/B, 32/24/C, 40/25/Z}
	\tkzDrawPoints(A, B, C)	
	\tkzDrawPoints[blue](Z)
	% vértices del triángulo
	\draw (A)--(B);
	\draw (B)--(C);
	\draw (C)--(A);
	\draw (C) node[above] {c};
	\draw (B) node[right] {b};
	\draw (A) node[left] {a};
	\draw (Z) node[right] {$z_0$};
	\end{tikzpicture}}
  \subfloat[CASO 1]{
  	\begin{tikzpicture}[scale=0.1]
	% puntos
	\tkzDefPoints{0/0/A, 50/0/B, 32/24/C,32/20/Z}
	\tkzDrawPoints(A, B, C)
	\tkzDrawPoints[blue](Z)

	% vértices del triángulo
	\draw (A)--(B);
	\draw (B)--(C);
	\draw (C)--(A);

	\draw[blue] (B)--(Z);
	\draw[blue] (Z)--(A);

	\draw (C) node[above] {$z_0$};
	\draw (B) node[right] {b};
	\draw (A) node[left] {a};
	\draw (Z) node[below] {$z'_0$};
	\end{tikzpicture}}\\

  \subfloat[CASO 2]{
    \begin{tikzpicture}[scale=0.1]
	% puntos
	\tkzDefPoints{0/0/A, 50/0/B, 32/24/C, 41/12/Z}
	\tkzDrawPoints(A, B, C)	
	\tkzDrawPoints[blue](Z)

	% vértices del triángulo
	\draw (A)--(B);
	\draw (B)--(C);
	\draw (C)--(A);
	\draw[blue] (A) --(Z);

	\draw (C) node[above] {c};
	\draw (B) node[right] {b};
	\draw (A) node[left] {a};
	\draw (Z) node[right] {$z_0$};
	\end{tikzpicture}}
  \subfloat[0.22][CASO 3]{
  	\begin{tikzpicture}[scale=0.1]
	% puntos
	\tkzDefPoints{0/0/A, 50/0/B, 32/24/C, 22/8/Z}
	\tkzDrawPoints(A, B, C)	
	\tkzDrawPoints[blue](Z)

	% vértices del triángulo
	\draw (A)--(B);
	\draw (B)--(C);
	\draw (C)--(A);
	\draw[blue] (A) --(Z);
	\draw[blue] (B) --(Z);
	\draw[blue] (C) --(Z);
	\draw (C) node[above] {c};
	\draw (B) node[right] {b};
	\draw (A) node[left] {a};
	\draw (Z) node[below] {$z_0$};
	\end{tikzpicture}}
\end{figure}

$\underline{CASO\ 0:}$
Si $z_0\notin \triangle$ se aplica el Teorema de Cauchy para triángulos.

$\underline{CASO\ 1:}$
Si $z_0$ es un vértice de $\triangle$ llamamos $\Gamma'$ a la frontera del triangulo $\triangle'$ con la misma base que $\triangle$ y con vértice $z_0'$ cercano a $z_0$.

Tomando $U' = U\backslash \{z_0\}$ y aplicando el Teorema de Cauchy para triángulos se tiene que $\int_{\Gamma'} f(z)dz =0$ Ahora bien, $\int_{\Gamma'} f(z)dz \xrightarrow[z_0'\to z_0]{} \int_{\Gamma} f(z)dz$ ya que $f$ es uniformemente continua en el compacto formado por el triangulo así que  $\int_{\Gamma} f(z)dz=0$.

$\underline{CASO\ 2:}$
Si $z_0$ está en un lado del triangulo, podemos subdividirlo en dos de forma que $z_0$ sea vértice de ambos, pudiendo aplicar el caso 1.

$\underline{CASO\ 3:}$
Si $z_0$ está en el interior del triangulo, subdividimos el triangulo en 3, de tal forma que $z_0$ sea vértice de los tres y aplicamos el caso 1.$\qed$
\end{dem}


\begin{theorem}[Teorema de Cauchy extendido para regiones estrelladas]
Sea $f(z)$ una función continua en un conjunto estrellado y abierto $U$ y  analítica en $U\backslash \{z_0\}$ para algún $z_0 \in U$. Para cualquier $\gamma$ camino cerrado en $U$ se tiene que $f(z)$ tiene primitiva y que:
$$
\int_{\gamma} f(z)dz = 0
$$
\end{theorem}
\begin{dem}
Análogo al teorema de Cauchy para regiones estrelladas pero usando el teorema de Cauchy extendido para triángulos.
\end{dem}


\begin{col}[Teorema de Cauchy para convexos]
Sea $U$ un conjunto abierto y convexo, $f(x)$ continua en $U$, analítica en $U\backslash \{z_0\}$ para algún $z_0 \in U$ entonces para todo $\gamma$ camino cerrado:
$$
\int_{\gamma} f(z)dz = 0
$$
\end{col}

\section{Formula integral de Cauchy para círculos}


\begin{theorem}[Formula integral de Cauchy para círculos]
Sea $f(z)$ analítica en un abierto $U$ que contiene a un disco $\overline{D}(z_0,r)$ para $r>0$. Entonces para cualquier $z \in D(z_0,r)$:
$$
f(z) = \frac{1}{2\pi i}\int_C \frac{f(w)}{w-z}dw
$$
Donde $C:= C(z_0,r) = \{z/ |z_0-z|=r\}$.
\end{theorem}

\begin{dem}
Sea $V = D(z_0,r)$ tal que $\bar{D}(z_0,r)\subset V\subset U$ y $z\in D(z_0,r)$. Sea $t>0$ tal que $D:= D(z,t)\subset D(Z_0,r)$.

(dibujo)

$C_1 \subset V_1$ estrellado entonces tenemos que $g(w) = \frac{f(w)}{w-z},\ w\in V_!$ es analítica, por lo que

$$\int_{C_1} \frac{f(w)}{w-z} dw = 0 \text{ y analogamente se tiene que } \int_{C_2} \frac{f(w)}{w-z} dw = 0$$
\begin{align*}
&0 = \int_{C_1} \frac{f(w)}{w-z} dw + \int_{C_2} \frac{f(w)}{w-z} dw = \int_{C} \frac{f(w)}{w-z} dw - \int_{D} \frac{f(w)}{w-z} dw\\
&\Rightarrow \int_{C} \frac{f(w)}{w-z} dw = \int_{D} \frac{f(w)}{w-z} dw = f(z) \int_C\frac{dw}{w-z}dz + \int_C\frac{f(w)-f(z)}{w-z}dw\\
&\int_{C} \frac{dw}{w-z} = \int_0^{2\pi} \frac{ite^{it}}{te^{it}}dt = 2\pi i
\end{align*}
$f(w)$ es continua en z con lo que $$\forall \varepsilon>0 \ \exists \delta>0 / |w-z|<\delta \Rightarrow f(w)-f(z)<\varepsilon .$$ Tomemos $0<t<\delta$
$$\abs{\int_D\frac{f(w)-f(z)}{w-z}dw} \leq  2\pi \varepsilon$$
Y como podemos hacer $\varepsilon$ arbitrariamente pequeño concluimos que 
$$
f(z) = \frac{1}{2\pi i}\int_C \frac{f(w)}{w-z}dw\ \qed
$$
\end{dem}


\begin{theorem}[Formula integral de Cauchy para círculos en derivadas]
Sea $f(z)$ analítica en un abierto $U$ que contiene a un disco $\overline{D}(z_0,r)$ para $r>0$. Entonces para cualquier $z \in D(z_0,r)\ \exists f^{(n)}(z),\ \forall n\in \N$ y se tiene que:
$$
f^{(n)}(z) = \frac{n!}{2\pi i}\int_C \frac{f(w)}{(w-z)^{n+1}}dw
$$
Donde $C:= C(z_0,r)$.
\end{theorem}

\begin{dem}
Sea $z_0\in D(z_0,r)$. Veamos que $f'(z) = \frac{1}{2\pi i}\int_C \frac{f(w)}{(w-z)^{2}}dw$.
Sea $h\neq 0$ suficientemente pequeño entonces:
\begin{align*}
&f(z) = \frac{1}{2\pi i}\int_C \frac{f(w)}{w-z}dw,\ f(z+h) = \frac{1}{2\pi i}\int_C \frac{f(w)}{w-(z+h)}dw\\
&\Rightarrow \frac{f(z+h)-´f(z)}{h}- \frac{1}{2\pi i}\int_C \frac{f(w)}{(w-z)^{2}}dw =\\
&= \frac{1}{2\pi ih}\int_C f(w)\left(\frac{1}{w-(z+h)}-\frac{1}{w-z}-\frac{h}{(w-z)^2}\right)dw=\\
&= \frac{1}{2\pi i h}\int_C \frac{f(w)h^2}{(w-z-h)(w-z)^2}dw = \frac{h}{2\pi i}\int_C \frac{f(w)}{(w-z-h)(w-z)^2}dw \xrightarrow[h\to 0]{} 0
\end{align*}

Veamos ahora que $f''(z) = \frac{2}{2\pi i}\int_C \frac{f(w)}{(w-z)^{3}}dw$.

\begin{align*}
&\frac{f'(z+h)-f'(z)}{h} - \frac{2}{2\pi i}\int_C \frac{f(w)}{(w-z)^{3}}dw =\\
&=\frac{1}{2\pi i h}\int_C f(w)\left(\frac{1}{(w-z-h)^2}-\frac{1}{(w-z)^2}-\frac{2h}{(w-z)^2}\right)dw =\\
&= \frac{1}{2\pi ih}\int_C f(w) \left(\frac{3h^2(w-z)-2h^2}{(w-z-h)^2(w-z)^3}\right)dw \xrightarrow[h\to o]{} 0
\end{align*}

Por inducción probemos ahora  que $f^{(n)}(z) = \frac{n!}{2\pi i}\int_C \frac{f(w)}{(w-z)^{n+1}}dw$:

\begin{align*}
&\frac{f^{(k)}(z+h)-f^{(k)}(z)}{h} - \frac{n!}{2\pi i}\int_C \frac{f(w)}{(w-z)^{n+1}}dw=\\
&= \frac{k!}{2\pi i h}\int_Cf(w)\left(\frac{1}{(w-z-h)^{k+1}}-\frac{1}{(w-z)^{k+1}}-\frac{(k+1)h}{(w-z)^{k+1}}\right)dw=\\
&= \frac{k!}{2\pi i h}\int_C f(w)\left(\frac{h^2(-{k+2 \choose 2} + (k+1){k+1\choose 1})(w-z)^{k}+h^3({k+1\choose 2} -\cdots)}{(w-z-h)^{k+1}(w-z)^{k+2}}\right)dw \xrightarrow[h\to o]{} 0
\end{align*}

Podemos concluir entonces que $f^{(n)}(z) = \frac{n!}{2\pi i}\int_C \frac{f(w)}{(w-z)^{n+1}}dw.\ \qed$
\end{dem}


\begin{col}[F.I.C]
Sea $z:0\in\C$ y $r>0$, $C:=C(z_0,r)$ Entonces:
$$
\int_C \frac{1}{w-z}dw = \left\{ \begin{array}{lcc}
             2\pi i &   si  & z\text{ es interior a } C \\
             \\ 0 & si & z\text{ es exterior a } C
             \end{array}\right.
$$
\end{col}

\begin{dem}
Si $z$ es exterior a $C$ entonces $f(w) = \frac{1}{w-z}$ es analítica en $D(z_0,s)$ con $s>r$ de forma que $z\notin D(z_0,s)$. Entonces por el Teorema de Cauchy para convexos $\int_C \frac{1}{w-z}dw = 0$.

Si $z$ es interior a $C$, $f(w)=1$ y por F.I.C entonces 
$$\frac{1}{2\pi i}\int_C \frac{1)}{w-z}dw = f(z) = 1 \Rightarrow \int_C \frac{1)}{w-z}dw = 2\pi i$$
\end{dem}



\begin{col}[Analiticidad de las derivadas]
Sea $f(z)$ una función analítica en un punto entonces sus derivadas de todos los ordenes son también funciones analíticas en ese punto.
Además si $f(z)$ tiene primitiva en un abierto $U$ entonces $f(z)$ es analítica en $U$.
\end{col}

\begin{dem}
Si $f$ es analítica en un punto $z_0$ entonces $\exists r>0$ tal que $f$ es derivable en todo punto de $D(z_0,r)$ y por F.I.C. para las derivadas $f$ es infinitamente derivable sobre $D(Z_0,r)$. Entonces todas las derivadas son analíticas en $z_0$.

Si $f(z)$ tiene primitiva en $U$ sabemos que $\exists F(z)$ analítica en $U$ tal que $F'(z) = f(z)\ \forall z\in U$.

Por lo anterior, todas las derivadas de $F$ e¡son analíticas en $U$, en  concreto $f(x)$ lo es.
\end{dem}

\begin{col}
Si una función$f(z) = u(x,y)+i(x,y)$ es analítica en un punto $z_0 = u_0 + iv_0$ entonces sus funciones componentes tienen derivadas parciales continuas de todo orden en ese punto.
Las partes reales e imaginarias de una función analítica en un dominio $D$ son funciones armónicas en $D$ 
\end{col}

\begin{col}
Sea $f(z)$ continua en un abierto $U$ y analítica en $U \backslash\{ z_0 \}$ para $z_0 \in U$ entonces $f(z)$ es analítica en $U$.
\end{col}

\begin{dem}
Sea $V=D(z_0,r),\ r>0$ tal que $V \subseteq U$. SI aplicamos el Teorema de Cauchy para convexos $\int_\gamma fZ)dz=0$ para todo camino cerrado en $V$. Entonces por el Teorema de independencia de caminos $\exists F(z)$ primitiva de $f$ en $V$. Con lo que $f(z)$ es analítica en $V$ y en consecuencia lo es también en $U$.
\end{dem}

\begin{defi}[Indice de un punto respecto a un camino cerrado]
$$n(\gamma,z_0) = \frac{1}{2\pi i}\int_\gamma \frac{dz}{z-z_0}$$
\end{defi}


\begin{theorem}[\textbf{Primer teorema de Cauchy}]
Sea $U$ abierto de $\C$ y $\gamma$ un camino cerrado, $\forall \f$ analítica, $$\int_\gamma f(z)dz = 0 \Leftrightarrow n(\gamma_z)=0 \ \forall z \notin U$$.
\end{theorem} 
\begin{theorem}[Segundo teorema de Cauchy]
Sea $U$ abierto de $\C$ son equivalentes:
\begin{enumerate}
\item $\overline{\C}\backslash U$ es conexo.
\item $n(\gamma,z)=0 \ \forall \gamma$ camino cerrado en $U$ y $\forall z \in \C\backslash U$.
\item $int_\gamma f(z)dz = 0 \ \forall f$ analítica en $U$ y $\forall \gamma$ camino cerrado en $U$.
\item Si $f$ es analítica entonces tiene primitiva.
\end{enumerate}
\end{theorem}


\begin{theorem}[Fórmula integral de Cauchy general]
Sea $f$ analítica en un abierto $U\in\C$ y $\gamma$ un camino cerrado en $U$ tal que $n(\gamma,z)=0$ $\forall z \notin U$ entonces para $z \in U$, $z \notin\gamma^*$ se cumple que:
$$f(z)n(\gamma,z) = \frac{1}{2\pi i}\int_\gamma \frac{f(w)}{z-z_0}dw$$
$$f^{(n)}(z)n(\gamma,z) = \frac{n!}{2\pi i}\int_\gamma \frac{f(w)}{(z-z_0)^{+1}}dw$$
\end{theorem}

\begin{defi}
Definimos los ciclos como sumas de la forma $\gamma = \sum_{j=1}^n a_i\gamma_i$ con $a_j \in\Z$ y $\gamma_j$ camino cerrado.
\end{defi}

\begin{theorem}
Sea $U$ abierto y $\gamma_1$ y $\gamma_2$ dos caminos cerrados (o ciclos) entonces
$$\int_{\gamma_1} f(z9dz = \int_{\gamma_2} f(z)dz\ \forall f \text{ analítica en }U \Leftrightarrow n(\gamma_1,z) = n(\gamma_2,z) \ \forall z\in \C\backslash U$$
\end{theorem}

\section{Aplicaciones de Teoría de Cauchy}

\begin{lema}
Sea $f(z)$ una función continua en $U$ abierto convexo, supongamos que para todo triangulo $\triangle \subset U$ se tiene que si $\int_\Gamma f(z)dz =0$ entonces $f(z)$ tiene primitiva.
\end{lema}

\begin{dem}
Sea $z_0 \in U$, definimos $F(z) = \int_{[z_0,z]} f(w)dw \ \forall z\in U$.

Tomemos $h \in \C$ tal que $|h| \ll |z_0-z|$ entonces:
\begin{multline*}
\abs{ \frac{F(z+h)-F(z)}{h}-f(z9} = \abs{ \frac{\int_{[z_0,z+h]}f(w)dw -\int_{[z_0,z]}f(w)dw}{h}-f(z)} =\\
 \int_{[z_0,z+h]}f(w)dw) + \int_{[z+h,z]}f(w)dw)+\int_{[z,z_0]}f(w)dw = 0\\
\Rightarrow \abs{ \frac{\int_{[z_0,z+h]}f(w)dw}{h} - \frac{\int_{[z,z+h]}f(w)dw}{h}} = \abs{ \frac{\int_{[z,z+h]}f(w)dw}{h}} \\
\leq |f(w)-f(z)| \xrightarrow[h\to 0]{} 0 \Rightarrow F'(z) = f(z) \qed
\end{multline*}
\end{dem}


\begin{theorem}[Teorema de Morera]
Sea $f(z)$ una función continua en n abierto $U$, supongamos que $\forall \triangle \subset U$ tal que $\int_\Gamma f(z)dz =0$ entonces $f(z)$ es analítica en $U$.
\end{theorem}

\begin{dem}
Sea $z_0\in U$ y $r>0$ consideramos $D(z_0,r)$. Por el lema anterior $f$ tiene primitiva en $D$ así que $f$ es analítica en $D$.

Como $z_0$ es arbitrario deducimos que $f$ es analítica en todo $U$.
\end{dem}

\begin{theorem}[Principio de reflexión de Schwarz]
Sea $f(z)$ una función analítica en el semiplano abierto $\C^+ = \{z\in\C / \Im(z) >0\}$ y continua en $\C^+ \cup \R$.

Supongamos que $\Im(f(z)) = 0\ \forall z\in \R$ entonces $f(z)$ puede ser extendida analíticamente a $\C$.
\end{theorem}

\begin{dem}
$$
f(z) = \left\{\begin{array}{cc} f(z) &z\in \C^+ \cup \R \\
						\overline{f}(\overline{z}) &z\in\C^-
		\end{array}\right.
$$$$
\lim_{h\to 0} \frac{\overline{f(\overline{z}+\overline{h})}-\overline{f(\overline{z})}}{h} = \lim_{h\to 0} \frac{\overline{f(\overline{z}+\overline{h})-f(\overline{z})}}{h} = \overline{f'(w)} 
$$
El límite existe así que es analítica $\qed$.
\end{dem}

\begin{lema}[Estimación de Cauchy]
Sea $f(z)$ analítica en $U \supset D(z_0,R)$ y $M_f(r)$ con $0<r<R$ entonces
$$\abs{ f^{(n)}(z_0} \leq \frac{n!}{r^n}M_f(r)\ \forall n = 0,1,\ldots$$
\end{lema}

\begin{dem}
Por F.I.C para las derivadas
\begin{multline*}
f^{(n)}(z) = \frac{n!}{2\pi i}\int_\gamma \frac{f(w)}{(z-z_0)^{+1}}dw \Rightarrow \abs{ f^{(n)}(z_0} = \abs{ \frac{n!}{2\pi i}\int_\gamma \frac{f(w)}{(z-z_0)^{+1}}dw }\\
 \leq \frac{n!}{\cancel{2\pi}}\cancel{2\pi} \max\left\lbrace\frac{|f(w)|}{|w-z_0|^{n+1}}, w\in C(z_0,r)\right\rbrace = \frac{n!}{r^n}M_f(z) \ \qed
\end{multline*}
\end{dem}


\begin{theorem}[Teorema de Liouville]
Si $f(z)$ es una función entera y está acotada entonces es constante.
\end{theorem}

\begin{dem}
Si $f(z)$ es acotada entonces $\exists M>0 / |f(z)|<M \ \forall z\in \C$. Por la estimación de Cauchy para $n=1$:
$$|f'(z)| \leq \frac{M_f(r)}{r}\leq \frac{M}{r} \xrightarrow[r\to \infty]{} 0 \ \forall r>0$$ Entonces $f'(z)=0 \ \forall z\in\C$ como $\C$ es conexo entonces $f$ es constante. $\qed$
\end{dem}


\begin{theorem}[Teorema fundamental del álgebra]
Todo polinomio $P(z)$ no constante tiene al menos una raíz, es decir, existe al menos un $z_0$ tal que $P(z_0)=0$.
\end{theorem}


\begin{dem}
Sea $P(z) = a_0 + a_1z + \cdots + a_n z^n, n>1$ Por R.abs., si $P(z)$ no tubiera raíces entonces $f(z) = \frac{1}{P(z)}$ es entera. Veamos que $f$ está acotada:

Sabemos que $\displaystyle |P(z)|\xrightarrow[|z|\to 0]{} 0 \Rightarrow \exists R>0 / |P(z)|>1 \text{ si } |z| >R \Rightarrow |f(z)| = \frac{1}{|P(z)|} <1$ si $|z|\leq R$. En consecuencia $f(z)$ es acotada en $\C$ y por el Teorema de Liouville $f(z) = cte.\ \sharp$
\end{dem}

\begin{col}[Teorema fundamental del álgebra]
Todo polinomio $P(z)$ de grado $n\geq 1$ con coeficientes complejos tiene exactamente $n$ raíces.
\end{col}


\begin{dem}
Por el Teorema anterior sabemos que existe $z_0$ tal que $P(z_0)=0$. Entonces podemos escribir $P(z) = (z-z_0)P_1(z)$ donde $P_1$ es un polinomio de grado $n-1$.

Si $n=1$ hemos acabado.

Si $n>1$ Aplicamos el proceso anterior a $P_1$, $P_1(z) = (z-z_1)P_2(z)$

Reiterando obtendremos que $P(z) = a_0(z-z1)(z-z_2)\cdots(z-z_n)$ 
$\qed$
\end{dem}


\begin{theorem}[Teorema de Gauss-Lucas]
Sea $P(z)$ un polinomio no constante con coeficientes complejos, los ceros de $P'(z)$ están en la clausura convexa de los ceros de $P(z)$.
\end{theorem}

\begin{dem}
$P(z)= \alpha \prod_{j=1}^n(z-a_j)$, con $n$ el grado de $P$ y $a_j$ sus raíces.

Sea $z_0\in\C$ tal que $P(z) \neq 0$ entonces 
$$\frac{P'(z_0)}{P(z_0)} = \sum_{j=1}^n \frac{1}{z_0-a_j}$$

Si $P'(z_0) = 0$  y $P(z_0)\neq 0$ entonces 
$$ \sum_{j=1}^n \frac{1}{z_0-a_j}=0 \Rightarrow \sum_{j=1}^n \frac{\bar{z_0}-\bar{a_j}}{|z_0-a_j|^2}=0 \Rightarrow z_0\sum_{j=1}^n \frac{1}{|z_0-a_j|^2} = a_j \sum_{j=1}^n \frac{1}{|z_0-a_j|^2}
$$
Entonces podemos escribir $z_0$ como $\sum_{j=0}^n \alpha_j a_j$ con $\sum_{j=0}^n \alpha_j = 1$, $\alpha_j >0$. $z_0$ es baricentro de $a_j$ y entonces $z_0$ está en la clausura convexa.

Si $P'(z_0) = 0$  y $P(z_0)= 0$ entonces $z_0 = 1\cdot z_0 + 0\cdot\sum a_j \Rightarrow z_0$ está en la clausura convexa. $\qed$
\end{dem}

\section{Principio del módulo máximo}

\begin{lema}[Propiedad del valor medio de Gauss]
Sea $f(z)$ analítica en $U\supset \bar{D(z_0,r)}$, $z_0\in\C$ y $r>0$ entonces
$$f(z_0) = \frac{1}{2\pi} \int_0^{2\pi} f(z_0+re^{it}dt$$
\end{lema}

\begin{dem}
$$f(z_0) = \frac{1}{2\pi i} \int_{C(z_0,r} \frac{f(w)}{w-z}dw,\ \gamma(t) = z_0+re^{it},\ \gamma'(t) = rie^{it}$$

$$
f(z_0) = \frac{1}{2\pi\cancel{i}}\int_0^{2\pi} \frac{f(z_0+re^{it})}{\cancel{r{e^{it}}}}\cancel{ri{e^{it}}}dt = \frac{1}{2\pi} \int_0^{2\pi} f(z_0+re^{it})dt
$$$\qed$
\end{dem}

\begin{theorem}[Principio del módulo máximo local]
Si $f(z)$ es analítica en un abierto $U$ y supongamos que $|f(z)|$ tiene máximo en $z_0\in U$. Entonces $f(z)$ es constante en un entorno de $z_0$.
\end{theorem}

\begin{dem}
Por hipótesis sabemos que $\exists D(z_0,r),\ r>0$ tal que $|f(z)| \leq |f(z_0)|\ \forall z \in D(z_0,r)$ Entonces se cumple que
$$
|f(z_0)| = \frac{1}{2\pi}\abs{\int_0^{2\pi}f(z_0+re^{it}dt} \leq \frac{1}{2\pi}\max\{|f(z_0+re^{it}/t\in [0,2\pi]\}\leq |f(z_0|
$$
$$
\Rightarrow |f(z_0)| = \frac{1}{2\pi}\int_0^{2\pi}|f(z_0+re^{it}|dt \Rightarrow \int_0^{2\pi} |f(z_0)|-|f(z_0+re^{it})|dt = 0
$$
$$
\Rightarrow |f(z_0) = |f(f(z_0+re^{it})|\ \forall t\in[0,2\pi]\Rightarrow |f(z_0)|= cte. \text{ en un entorno de } z_0.
$$$\qed$
\end{dem}

\begin{col}
Sea $f(z_0)$ analítica en un abierto y conexo $U$, supongamos que $|f(z_0)|$ alcanza el máximo absoluto en $U$, entonces $f(z_0)$ tiene valor constante en $U$.
\end{col}
\begin{dem}
Por hipótesis $\exists z_0 \in U / |f(z)|\leq |f(z_0)| \ \forall z \in U$.

Sea $E = \{z\in U / f(z) = f(z_0)\}$ al ser $f(z)$ continua en $U$ claramente el conjunto $E$ es cerrado respecto de $U$  y $z_0\in E \Rightarrow E \neq \emptyset$ .

Sea $z\in E \Rightarrow |f(z)| = |f(z_0)| = \max\{|f(z)|\}$ entonces por el resultdo anterior $f$ es constante en un entorno de $z$. Con lo que $\exists\  r>0 $tal que $D(z,r)\subset E \Rightarrow E$ es abierto y entonces $E=U$. $\qed$
\end{dem}

\begin{theorem}[Principio del módulo máximo]
Sea $f(z)$ analítica en un conjunto abierto, conexo y acotado $U$ y continua en $Fr(U)$. Si $M =\max \{|f(z)|/ z\in Fr(U)\}$, se cumple que:
\begin{enumerate}
\item $|f(z)|\leq M \ \forall z \in U$
\item Si $|f(z)| = M$ para $z_0 \in U$ entonces $f(Z) = cte.$ en $U$.
\end{enumerate}
\end{theorem}

\begin{dem}
\begin{enumerate}
\item Observamos que $|f(z)|$ es también continua en $\bar{U}$ por lo que $|f(z)|$ alcanza un máximo en $\bar{U}$. Si tal máximo se alcanza en $Fr(U)$ hemos terminado.
Si el máimo se alcanza en  $Int(U)$ entonces $\exists z_0 \in U$ tal que $|f(z)|\leq |f(z_0)| \ \forall z\in U$ entonces $f$ es constante en $U$.
\item Si $\exists z_0 \in U$ tal que $|f(z_0)| = M$ entonces $f(z_0)$ es constante.$\qed$
\end{enumerate}
\end{dem}


\begin{theorem}[Principio del módulo mínimo]
Sea $f(z)$ analítica en un conjunto abierto, conexo y acotado $U$ y continua en $Fr(U)$.Supongamos $f(z)\neq 0\ \forall z\in U$. Si $m =\min \{|f(z)|/ z\in Fr(U)\}$, se cumple que:
\begin{enumerate}
\item $|f(z)|\geq n \ \forall z \in U$
\item Si $|f(z)| = n$ para $z_0 \in U$ entonces $f(Z) = cte.$ en $U$.
\end{enumerate}
\end{theorem}

\begin{dem}
Sea $g(z)= \frac{1}{f(z)}$ analítica en $U$ y continua en $\bar U$. Vemos que $g$ cumple las condiciones del módulo máximo con lo que el máximo de $g$ se alcanza en la frontera y en consecuencia el mínimo de $f$ también.$\qed$
\end{dem}

\begin{col}
Sea $f(z)$ analítica no constante en un abierto conexo y acotado $U$ y continua en $Fr(U)$ . Si $U(x,y) = \Re(f(z))$ y $v(x,y) = \Im(f(z))$ entonces $u,v$ satisfacen el principio del módulo máximo (mínimo).
\end{col}

\begin{dem}
Sea $g(z)= e^{f(z)}$ analítica en $U$ y continua en la frontera. $|g(z)| = |e^{f(z)}| = |e^{\Re(f(z)}||e^{\Im{f(z)}i}| = |e^u||e^{iv}| = e^u$ como $g$ satisface el principio del módulo máximo (mínimo) $u$ también los cumple.

Sea  $ h(z) = e^{-if(z)}$ entonces $|h(z)| = e^v$ y como $h$ satisface el principio del módulo máximo (mínimo) $v$ también los cumple.$\qed$
\end{dem}

\begin{lema}[Lema de Schwarz]
Sea $f(z)$ una función analítica en $D(0,1)$. Supongamos que $f(0)=0$ y $|f(z)|\leq 1\ \forall z \in D(0,1)$ entonces $|f(z)| \leq |z|$ y $|f'(0)| \leq 1$.

Además si $\exists z_0 \in D(0,1)\backslash \{0\}$ tal que $|f(z_0)| = |z_0|$ o que $|f'(z_0)| = 1$ entonces $f(z) = az$  $\forall z\in D(0,1)$ , $as \in \C/ |a|=1$.
\end{lema}

\begin{dem}
$$ \text{Sea } g(z) = \left\lbrace \begin{array}{ccc}
						\frac{f(z)}{z}& si & z\neq 0\\
						f'(0) & si & z=0\end{array}\right.$$

Si $z\neq 0$ $g$ es analítica. 

Si $z=0$ entonces $g$ es derivable porque 
$$\exists \lim_{h\to 0} \frac{g(h)-g(0)}{h} = \lim_{h\to 0} \frac{\frac{f(h)}{h}-f'(0)}{h}= \lim_{h\to 0} \frac{f(h)-hf'(0)}{h^2}$$
El límite existe por analiticidad de $f$.

De hecho es claro que $g$ es continua en $z=0$ y por tanto $g$ es analítica en $D(0,)$.

Por el P.M.Máximo $|g(z) |\leq M = \max\{|g(z)|/ z\in FR(U)\}$, $|g(z)| = \frac{|f(z)|}{z}\leq 1 \Rightarrow |f(z)| \leq z \ \forall z\in D(0,1)$.

La segunda parte de la afirmación se cumple por el apartado del P.M.-mínimo.
\end{dem}

\begin{defi}
Llamaremos núcleo de Poisson a la función 
$$P_r(x) = \frac{R^2-r^2}{R^2+r^2-2rR\cos(x)} \text{ siendo } 0\leq r < R \text{ y } x\in\R$$ 
\end{defi}

\begin{defi}
Llamaremos núcleo de Cauchy a la función 
$$Q_z(t) = \frac{Re^{it}+z}{Re^{it}-z} \text{ siendo } z\in D(0,R), t\in\R$$ 
\end{defi}
\textbf{Resultado}: $\Re(Q_z(t)) = P_r(\theta-t)$ con $z= re^{it}\in D(0,R)$ y $0\leq \theta < 2\pi$.



\begin{theorem}[Fórmula integral de Poisson]
Sea $f(z)$ analítica en $D(z_0,R)$ y continua en $\overline{D(z_0,R)}$, para $z_0\in\C,\ R>0,\ z= z_0+e^{it}\in D(z_0,R)$, $0\leq r<R$ se tiene que:
$$f(z) = \frac{1}{2\pi}\int_0^{2\pi} P_r/\theta-t)f(z_0+Re^{it})dt$$

Si $u = \Re(f)$ entonces $$u(z) = \frac{1}{2\pi}\int_0^{2\pi} P_r/\theta-t)u(z_0+Re^{it})dt$$
\end{theorem}


\chapter{Series de potencias}
\section{Convergencia de Series}
\begin{defi}
Una sucesión de números complejos $\{z_n\}_{n>=1}$ es convergente a $z\in\C$ si $\forall\varepsilon>0 \ \exists n_0\in\N / |z_n-z|<\varepsilon\ \text{ si } n>n_0$.
\end{defi}

\begin{defi}
Dada una serie de números complejos $\sum_{n\geq 1} z_n$ diremos que:

$a)$ Converge a $z$ si $\forall \varepsilon>0\ \exists n_0\in\N / \abs{\sum_{n=1}^mz_n-z} <\varepsilon$ cuando $m> n_0$.

$b)$ Converge absolutamente si $\sum_{n\geq 1} |z_n| < \infty$.
\end{defi}
\textbf{Observación:}
\begin{enumerate}
\item $\sum_{n\geq 1} z_n$ converge si la sucesión de sumas parciales $\{\sum_{k=1}^n z_k\}_{n\geq 1}$ converge, es decir, $\exists \lim_{n\to infty}\sum_{k=1}^n z_k$.
\item Por la completitud de $\C$ equivale a que $\{\sum_{k=1}^n z_k\}_{n\geq 1}$ es de Cauchy.
\item Si $z_n = x_n + iy_n$ entonces $\sum_{n\geq 1} z_n$ converge si $\sum_{n\geq 1} x_n < \infty$  y $\sum_{n\geq 1} y_n<\infty$.
\item Una condición necesaria para la convergencia de la serie $\sum_{n\geq 1} z_n$ es que $\lim_{n\to \infty} z_n = 0$.
\item Convergencia absoluta implica convergencia.
\end{enumerate}

\subsection{Criterios de convergencia para series}

\begin{enumerate}
\item Algunos criterios en $\R$: comparación, cociente y raíz funcionan para $\C$.
\item Dada la serie $\sum_{n\geq 1} P_n$, $P_n\geq 0$, si consideramos $A = \limsup(P_n)^{\frac{1}{n}}$ y $B = \limsup\frac{P_{n+1}}{P_n}$ se tiene que:
	\begin{enumerate}
	\item Si $A <1$ o $B<1$ entonces $\sum_{n\geq 1} P_n< \infty$
	\item Si $A >1$ o $B>1$ entonces $\sum_{n\geq 1} P_n= \infty$
	\end{enumerate}
\end{enumerate}


\section{Series de potencias}

\begin{defi}
Sea $z_0\in \C$ y $\{a_n\}_{n\geq 0}$ una sucesión en $\C$. La serie de potencias de coeficientes $\{a_n\}$ y centro $z_0$ es la serie funcional dada por
$$
\sum_{n\geq 0} a_n(z-z_0)^n
$$ 
\end{defi}

\begin{defi}
Dada una serie de potencias $\sum_{n\geq 0} a_n(z-z_0)^n$, diremos que su radio de convergencia viene dado por:
$$
r = \left(\limsup_{n\to \infty} |a_n|^{\frac{1}{n}}\right)^{-1}
$$
\end{defi}

\begin{defi}
Dada una serie de potencias de números complejos $\sum_{n\geq 0} a_n(z-z_0)^n$, diremos que:
\begin{enumerate}
\item Converge en un punto $w_0$ si $\sum_{n\geq 0} a_n(w_0-z_0)^n$ converge. En caso contrario diremos que no converge a $w_0$.
\item Converge absolutamente en un punto $w_0$ si $\sum_{n\geq 0} \abs{a_n(w_0-z_0)^n}<\infty$.
\item Converge uniformemente sobre un conjunto $S \subset \C$ a una función $f(w)$ si 
$$
\forall\varepsilon>0,\ \exists n_0\in\N /\ \abs{\sum_{n\geq 0} a_n(w-z_0)^n-f(w)}<\varepsilon \ \forall n> n_0,\ \forall w\in S
$$

La elección de $n_0$ sólo depende del valor del valor de $\varepsilon$ y es independiente del punto $w$ que se tome de $S$.
\end{enumerate}
\end{defi}


\begin{theorem}[Teorema de Abel]
Dada una serie de potencias de números complejos  $\sum_{n\geq 0} a_n(z-z_0)^n$, supongamos que converge para $z_1\in\C$ y llamaremos $r = |z_1-z_0|$ entonces la serie converge absolutamente en todo $D(z_0,r)$ y uniformemente en todo compacto de $D(z_0,r)$ a la función suma $f(z) =  \sum_{n\geq 0} a_n(z-z_0)^n$.
\end{theorem}

\begin{dem}
Sea $z\in D/z_0,r)$ entonces  $\sum_{n\geq 0} \abs{a_n(z-z_0)^n} = \sum_{n\geq 0}\abs{a_n\frac{(z-z_0)^n}{(z_1-z_0)^n}(z_1-z_0)^n}$.

Dado que  $\sum_{n\geq 0} a_n(z_1-z_0)^n$ converge entonces $\lim_{n\to \infty} a_n(z_1-z_0)^n =0$ y $\exists M>0 / |a_n(z_1-z_0)^n|<M\ \forall n\in\N$.

$\Rightarrow \sum_{n\geq 0} \abs{a_n(z_1-z_0)^n} \leq M\sum_{n\geq 0} \abs{\frac{z-z_0}{z_1-z_0}}^n$ que converge por ser serie geométrica de razón menor que 1.

Tomemos ahora $k\subset D(z_0,r)$ un conjunto compacto, y sea $z' \in D(z_0,r)$ talque $k\subset \bar{D}(z_0,r')\subset D(z_0,r)$ con $r' = |z'-z_0|$.
Así si $z\in k$ tenemos que $\sum_{n\geq 0} |a_n(z-z_0)^n| <\infty \Rightarrow \sum_{n\geq 0} a_n(z-z_0)^n < \infty\ \forall z \in K$.

Sea ahora $N_1 \in \N$ y $w \in k$ entonces:
\begin{align*}
&\abs{ \sum_{j=0}^{N_1-1} a_j(w-z_o)^j - \sum_{j\geq 0}a_j(w-z_0)^j } = \abs{ \sum_{j\geq N_1}a_j(w-z_0)^j  } =\\
&=\lim_{N_2 to \infty} \abs{ \sum_{j = N_1}^{N_2}a_j(w-z_0)^j  } \leq \lim_{N_2\to \infty} \sum_{j=N_1}^{N_2}|a_j||w-z_0|^j \leq\\
&\leq lim_{N_2 \to infty} \sum_{j=N_1}^{N_2}|a_j||z'-z_0|^j \leq \sum_{n\geq 0}|a_n||z'-z_0|^n < \infty
\end{align*}
Los restos de la serie cumplen la definición de convergencia uniforme.$\qed$
\end{dem}

\begin{theorem}
Consideremos la serie de potencias $\sum_{n\geq 0} a_n(z-z_0)^n$ y $r$ su radio de convergencia, consideremos el disco $D := F(z_0,r)$.
\begin{enumerate}
\item La serie converge absolutamente $\forall z \in D$.
\item La serie converge uniformemente en todo compacto de $D$.
\item La serie no converge sea cual sea $z$ tal que $|z-z_0|>r$.
\end{enumerate}
\end{theorem}


\begin{dem}
Recordemos que $r = \left(\limsup|a_n|^{\frac{1}{n}}\right)^{-1}$.
\begin{enumerate} 
\item Sea $z\in  D$, observemos que 
$$\limsup|a_n(z-z_0)^n|^{\frac{1}{n}} = \limsup|a_n|^{\frac{1}{n}}|z-z_0| = |z-z_0|r^{-1} = \frac{|z-z_0|}{r}< 1 $$
Por criterio de la raíz hay convergencia absoluta con $z\in D$.

\item La convergencia uniforme en todo compacto de $D$ se deduce por el teorema de Abel.

\item Por reducción al absurdo, tomemos que la serie converge para algún $z' / |z'-z_0|>r$.

Por el teorema de Abel la serie también convergería en $z\in D(z_0,r')$ con $r? = |z'-z_0|$.

En particular converge $\forall z / |z-z_0 <r'$, es decir, también converge en $z/ r<|z-z_0|z|'-z_0|$ lo que supone una contradicción ya que 
$$\limsup|a_n(z-z_0)^n|^{\frac{1}{n}} = \frac{z-z_0|}{r}>1\ \sharp$$
\end{enumerate}
\end{dem}


\section{Teorema de Taylor y de la convergencia analítica}


\begin{theorem}[Taylor]
Sea $f(z)$ analítica en $D(z_0,r)$ para $z_0\in\C$ t $r>0$ entonces para todo $z \in D(z_0,r)$, $f(Z)$ admite la representación en serie:
$$f(z) = \sum_{n\geq 0} \frac{f^{(n)}(z_0)}{n!}(z-z_0)^n$$
\end{theorem}

\begin{dem}
Sea $r_1 /  0<r_1<r$, $C_1 := \{z/|z-z_0|=r_1\},\ z\in D(z_0,r_1)$ entonces:
\begin{multline*}
f(z) = \frac{1}{2\pi i} \int_{C_1}\frac{f(w)}{w-z}dw = \frac{1}{2\pi i}\int_{C_1} \frac{f(w)}{w-z-(z-z_0)}dw =\\
= \frac{1}{2\pi io} \int_{C_1} \frac{f(w)}{(w-z_0)\left(1-\frac{z-z_0}{w-z_0}\right)}dw
\end{multline*}

Para $s\neq 1$ tenemos que $\frac{1}{1-s} = 1 +s+s^2+\cdots s^{n-1} +\frac{s^n}{1-s}$
\begin{align*}
&\Rightarrow \frac{1}{1-\frac{z-z_0}{w-w_0}} = 1 + \frac{z-z_0}{w-z_0} +\frac{(z-z_0)^2}{(w-z_0)^2} +\cdots + \frac{(z-z_0)^{n-1}}{(w-z_0)^{n-1}} + \frac{\left(\frac{z-z_0}{w-w_0}\right)^n}{1-\frac{z-z_0}{w-w_0}}\\
&\Rightarrow \frac{1}{(w-z_0)\left(1-\frac{z-z_0}{w-w_0}\right)} = \frac{1}{w} +  \cdots + \frac{1}{(w-z_0)^n}(z-z_0)^{n-1} + \frac{1}{(w-z)(w-z_0)^{n+1}}(z-z_0)^n\\
&\text{por F.C.D. } \frac{1}{2\pi i} \int_{C_1}\frac{f(w)}{(w-z_0)^{n+1}}dw = \frac{f^{(n)}(z_0)}{n!}\\
& \Rightarrow f(z) = f(z_0) + \frac{f'(z_0)}{1!}(z-z_0) +\frac{f''(z_0)}{2!}(z-z_0)^2+ \cdots + \frac{f^{(n-1)}(z_0)}{(n-1)!}(z-z_0)^{n-1} + P_n(z)\\
& \text{con } P_n(z) = \frac{(z-z_0)^n}{2\pi i}\int_{C_1} \frac{f(w)}{w-z)(w-z_0)^{n+1}}dw
\end{align*}

Para concluir solo queda demostrar que $|P_n(z)| \xrightarrow[n\to \infty]{}0$.

Tomemos $M = \max_{z\in C_1}\{|f(w)|\}$, $|z-z_0| = r_2 < r_1$, $|w-z| = |w-z_0-(z-z_0)| \geq r_1-r_2$, tenemos que
$$|P_n(z)| \leq \frac{r_2^n}{2\pi}\frac{M}{r_1^{n+1}(r_0-r_1}2\pi r_1 = \frac{M}{r_1-r_2}\left(\frac{r_2}{r_1}\right)^n \xrightarrow[n\to \infty]{}0$$
$$\Rightarrow f(z) = \sum_{n\geq 0} \frac{f^{(n)}(z_0)}{n!}(z-z_0)^n\ \qed$$ 
\end{dem}


\begin{theorem}[Convergencia analítica]
Sea $\{f_n(z)\}_n$ una sucesión de funciones analíticas en un abierto $U$ tal que $f_n(z) \to f(z)$ uniformemente sobre los compactos  de $U$, entonces $f(z)$ es analítica en $U$ y $f_n^{(p)}(z)$ converge uniformemente a $f^{(p)}(z)$ sobre los compactos de $U$.
\end{theorem}

\begin{dem}
Veamos que $f(z)$ es continua:
$$\text{Sea } \varepsilon >0,\ N \in \N\ / |f_n(z)-f(z)| < \varepsilon/3\ \forall n> N,\ \forall z\in U$$
$$\text{Sea } z_0 \in U \Rightarrow \exists \delta >0\ /\ |f_N(z)-f_N(z_0)|<\varepsilon/3 \text{ si }|z-z_0|<\delta$$
$$ \Rightarrow |f(z)-f(z_0)| \leq |f(z)-f_N(z)|+|f_N(z9-f_N(z_0)|+|f_N(z_0)-f(z_0)|<varepsilon$$
Entonces $f(z)$ es continua en $z_0$. Y si tomamos $k$ un compacto de $U$ y $\Gamma$ la frontera de $\triangle$ tenemos que
$$\int_\Gamma f(z)dz = \lim_{n\to \infty}\int_\Gamma f_n(z)dz = 0 \xRightarrow{T^a\ Morera} f(z) \text{ es analítica en } k\Rightarrow \text{ es analítica en } U.$$

Sea $r>0$ y $z_0\in U$ tal que $\overline{D(z_0,r)}\subset U$ y $C := C(z_0,r)$ entonces
$$f_n^{(p)}(z) - f^{(p)}(z) = \frac{p!}{2\pi i}\int_C \frac{f_n(w)-f(w)}{(w-z)^{p+1}}dw,\ \forall z\in D(z_0,r)$$

Sea $\varepsilon >0$ y $N_1 \in \N$: $f_n(z)$ es continua uniformemente en $\overline{D(z_0,r_1)}$ con $0<r_1<r$ entonces  se cumple que $\forall z \in\overline{D(z_0,r_1)}$
$$\abs{f_n^{(p)}(z) - f^{(p)}(z)} \leq \frac{p!}{2\pi}\frac{\varepsilon}{(r-r_1)^{p+1}}\ \forall n>N$$
Entonces $f_n^{(p)}(z)$ tiende a $f^{(p)}(z)$ sobre los compactos de $U$. $\qed$
\end{dem}

\begin{defi}[Serie de Taylor]
Si $f(z)$ es analítica en $D(z_0,r)$ para algún $z_0\in\C$ y $r>0$ entonces
$$f(z) = \sum_{n\geq 0} \frac{f^{(n)}(z_0)}{n!}(z-z_0)^n$$
se llama serie de Taylor de $f$ centrada en $z_0$. Si $z_0=0$ se llama serie de MacLaurin. 
\end{defi}


\begin{theorem}[Recíproco del Teorema de Taylor]
Sea $f(z)$ una función definida en $D(z_0,r)$ para algún $z_0\in \C$ y $r>0$, si $f(z)$ admite un desarrollo en serie de potencias $\sum_{n\geq 0} a_n(z-z_0)^n$ para todo $z\in D(z_0,r)$ entonces $f$ es analítica en $D(z_0,r)$.
\end{theorem}

\begin{dem}
Es claro que el radio de convergencia de la serie es mayor o igual que $r$, entonces la serie converge uniformemente sobre compactos de $D(z_0,r)$.con lo que si tomamos $f_n(z) = \sum_{k= 0}^n a_k(z-z_0)^k$ vemos como el límite uniforme de $f_n(z)$ es $f(z)$ y por el teorema de convergencia analítica $f$ es analítica en $D(z_0,r)$. $\qed$
\end{dem}



\begin{theorem}[Unicidad de coeficientes]
Sea $f(z)  = \sum_{n\geq 0} a_n(z-z_0)^n$ una función definida en $D(z_0,r)$ para algún $z_0\in \C$ y $r>0$, entonces
\begin{enumerate}
\item $a_n = \frac{f^{(n)}(z_0)}{n!}$
\item $f(z)$ es indefinidamente derivable en  $D(z_0,r)$ y $f^{(p)}(z) = \sum_{n\geq 0}a_nn(n-1)\ldots (n-p+1)(z-z_0)^{n-p},\ p\in \N$.
\end{enumerate}
\end{theorem}

\begin{dem}
Dado que $f(z) = \sum_{n\geq 0} a_n(z-z_0)^n$ converge uniformemente en todo compacto de $D(z_0,r)$, por el teorema de convergencia analítica $\{fN^{(p)}\}$ converge uniformemente en todo compacto de $D(z_0,r)$a $f^{(p)}(z_0)$.

Ahora para probar 1, hacemos $z=z_0$ en la igualdad de 2:
\begin{align*}
f^{(p)}(z_0) = \sum_{n\geq 0}a_nn(n-1)\ldots (n-p+1)(z_0-z_0)^{n-p}\\
\Rightarrow f^{(p)}(z_0)= a_p p!\Rightarrow a_p = \frac{f^{(p)}}{p!}\ p\in \N
\end{align*}
\end{dem}

\section{Principio de prolongación analítica}

\begin{defi}
Sea $f(z)$ una función definida en $U$ entonces el conjunto de ceros de $f(z)$ se denota como $Z(f) = \{z\in U / f(Z) = 0\}$.

Si $f(z)$ es analítica en $z_0$, diremos que $f(z)$ tiene un cero de orden $m\geq 1$ si $f(z) = \sum_{n\geq0} a_n(z-z_0)^n$ con $a_1 = a_2 = \cdots=a_{m-1} = 0$ y $a_m \neq 0$.

También de forma equivalente $f(z)$ tiene un cero en $z_0$ de orden $m\geq 1$ cuando $f(Z) = (z-z_0)^m g(z)$ con $g(z)$ analítica en $z_0$ y $g(z_0)\neq 0$.
\end{defi}

\begin{lema}
Sea $f(z)$ analítica en un abierto $U$ y denotamos por $A$ el conjunto d los puntos de acumulación de $Z(f)$ en $U$, entonces $A$ es abierto y cerrado en $U$.
\end{lema}


\begin{dem}
$$A = \{z\in U /\ \lim z_n = z,\ z_n\in Z(f),\ z_n\neq z_m\}$$
Observamos en primer lugar que $z\in A$ implica que $z\in Z(f)$.

Veamos que $A$ es cerrado. Consideramos $\{w_n\}\subset A$ tal que $\{w_n\}_n\to w$ y probemos que $w \in A$. Esto es cierto ya que $w_n\in Z(f)$ y $w_n\neq w\ \forall n\in\N$.

Veamos que $A$ es abierto, viendo que todo punto de $A$ es interior. Sea $z_0\in A$ y consideramos $f(z) = \sum_n a_n (z-z_0)^n$ válido para $D(z_0,r)$,$r>0$ entonces $Dz_0,r)\subset U$.

Como $z_0 \in A \Rightarrow z_0\in Z(f),\ f(z_0) = 0$ y supongamos que $a_0 = a_1 = \cdots=a_{m-1} = 0$ y $ a_m\neq 0$, si $f(Z) = (z-z_0)^m g(z)$ con $g$ analítica y $g(z_0)\neq 0$.

Entonces $z_0$ no es un punto aislado de $Z(f)$. $\sharp$

Concluimos entonces que $a_n = 0 \forall n\in N\Rightarrow f(z)=0\ \forall z in  D(z_0,r)\subset A \Rightarrow A$ es abierto. $\qed$
\end{dem}


\begin{theorem}[Principio de identidad]
Sea $f(z)$ analítica en un conjunto abierto y conexo $U$. Supongamos que  existe una sucesión $\{z_n\}_{n\geq 1}$ de puntos distintos en $U$ tal que $\exists \lim_{n\to \infty}z_n = z_0 \in U$ con $f(z_n)=0$, $\forall n \in \N$.  Entonces $f(z) =0 \forall z \in U$.
\end{theorem}

\begin{dem}
Sea $A = \{z\in U /\ \lim z_n = z,\ z_n\in Z(f),\ z_n\neq z_m\}$ entonces $A\neq \emptyset$ y por el lema anterior $A = U$ y entonces $f(z) = 0\ \forall z\in U$.
\end{dem}


\begin{col}[Principio de prolongación analítica]
Sean $f(z)$ y $g(z)$ analíticas en un abierto $U$ y conexo.

Si $f(z) = g(z)$ $\forall z\in S$ y $S\subset U$ es un conjunto con al menos un punto de acumulación de $U$ entonces $f(z) = g(z)\ \forall z\in U$.
\end{col}

\begin{dem}
Tomemos $h(z) = f(z)-g(z)$, analítica en $U$.

Vemos que $h(z) = 0\ \forall z \in S$ con lo que $Z(f)$ tiene punto de acumulación en $S$.

Y por el teorema anterior $h(z)= \ \forall z\in U$, es decir, $f(z) = g(z) \ \forall z \in U$. $\qed$.
\end{dem}

\section{Series de Laurent}
\begin{defi}
DEcimos que $f(z)$ tiene una singularidad en $z_0$ si $f(z)$ no es analítica en $z_0$ pero sí loes en algún punto de todo entorno de $z_0$.

Esta singularidad se dice que es aislada si existe un entorno perforado de $z_0$ donde $f(z)$ es analítica. En caso contrario diremos que $z_0$ es una singularidad no aislada.
\end{defi}

\begin{lema}
Sea $f(z)$ analítica en un abierto $U$, con $\{z\in\C / r_1 \leq |z-z_0|\leq r_2\} \subset U$. Consideremos $C_j = \{z\in\C / |z-z_0| = r_j\}$, $j= 1,2$
Entonces $\forall z/ r_1 < |z-z_0| < r_2$ tenemos qe 
$$f(z) = \frac{1}{2\pi i} \int_{C_2} \frac{f(w)}{w-z}dw - \frac{1}{2\pi i} \int_{C_1} \frac{f(w)}{w-z}dw$$
\end{lema}

\begin{dem}
Sea $\gamma = C_2 -C_1$ un ciclo.

Observamos que si $z_1 \notin U$ y $|z_1-z_0|> r_2$ entonces $n(\gamma,z_1) = n(C_2,z_1) - n(C_1,z_1) = 0$. Y si $z_1 \notin U$ y $|z_1-z_0| <r_1$ entonces $n(\gamma,z_1) = 1-1 = 0$.

Con lo que $n(\gamma,z) = 0 \ \forall z\notin U$ y por F.I.C.:
$$n(\gamma,z)f(z) = \frac{1}{2\pi i} \int_{\gamma} \frac{f(w)}{w-z}dw = \frac{1}{2\pi i} \int_{C_2} \frac{f(w)}{w-z}dw - \frac{1}{2\pi i} \int_{C_1} \frac{f(w)}{w-z}dw$$
Y para cualquier $z_0$ en la corona se tiene que $n(\gamma,z_0) = n(C_2,z_0) - n(C_1,z_0)=1-0 = 1$. $\qed$
\end{dem}


\begin{theorem}[Teorema de Laurent]
Sea $f(z)$ analítica en la corona $U = \{z\in\C / S_1 \leq |z-z_0|\leq S_2\}$ con $0\leq S1\leq S2\leq \infty$ y algún $z_0 \in \C$. Para todo $z\in U$ se tiene que:
$$f(z) = \sum_{n=-\infty}^{infty} an(z-z_0)^n,\ a_n = \frac{1}{2\pi i}\int_C \frac{f(w)}{(w-z)^{n+1}}dw$$
Con $C = (C(z_0,r)$ y $S_1 < r< S2$.
\end{theorem}

\begin{dem}
Consideremos $C_j = \{z\in\C / |z-z_0| = r_j\}$, $j= 1,2$ con $S_1 < r_1 < r_2 <S_2$.
Entonces $\forall z / |z-z_0|<r$ tenemos 	que 
$$f(z) = \frac{1}{2\pi i} \int_{C_2} \frac{f(w)}{w-z}dw - \frac{1}{2\pi i} \int_{C_1} \frac{f(w)}{w-z}dw$$
y análogo al teorema de Taylor:

$$\frac{1}{2\pi i} \int_{C_2} \frac{f(w)}{w-z}dw = \sum_{n\geq 0}(z-z_0)^n \frac{1}{2\pi i} \int_{C_2} \frac{f(w)}{(w-z)^{n+1}}dw$$

Obtenemos $a_n = \frac{1}{2\pi i}\int_C \frac{f(w)}{(w-z)^{n+1}}dw$.

Por el teorema de Abel $f$ converge absolutamente en $D(z_0,r_2)$ y uniformemente sobre compactos de $D$. Dado que $\abs{\frac{w-z_0}{z-z_0}}<1$ aplicamos lo mismo para $C_1$:
$$-\frac{1}{2\pi i} \int_{C_1} \frac{f(w)}{w-z}dw = \sum_{n\geq 0}(z-z_0)^n \frac{1}{2\pi i} \int_{C_2} \frac{f(w)}{(w-z)^{n+1}}dw$$
Obtenemos $b_n = -\frac{1}{2\pi i}\int_C \frac{f(w)}{(w-z)^{n+1}}dw$.

De esta forma podemos escribir $f(z)$ como $\sum_{n\geq 0} a_n (z-z_0)^n +  \sum_{n\geq 0} b_n (z-z_0)^n$ con lo que 

$$f(z) = \sum_{n\geq 0} a_n (z-z_0)^n +  \sum_{n= -\infty}^{-1} a_n (z-z_0)^n\ \forall z\in U$$$\qed$
\end{dem}

\begin{prop}
Sea $z_0$ una singularidad aislada de $f(z)$ entonces 
\begin{enumerate}
\item $z_0$ es una singularidad evitable $\Leftrightarrow \lim_{z\to z_0} f(z) \in\C$.
\item $z_0$ es un polo de orden $m \Leftrightarrow \lim_{z\to z_0} f(z)(z-z_0)^m \in \C\backslash\{0\}$
\end{enumerate}
\end{prop}

\begin{dem}
\begin{enumerate}
\item $$z_0 \text{ es evitable } \Leftrightarrow f(z) = \sum_{n=0}^{\infty} a_n (z-z_0)^n \Leftrightarrow \lim_{z\to z_0} f(z) = lim_{z\to z_0} \sum_{n=0}^\infty a_n (z-z_0)^n = a_0$$

\item \begin{align*}
& z_0 \text{ es polo de orden } m \Leftrightarrow f(z) = \sum_{n=-m}^{-1} a_n(z-z_0)^n + \sum_{n=0}^\infty a_n (z-z_0)^n\\
& \Leftrightarrow g(z) = f(z)(z-z_0)^m = a_{-m} + aa_{m-1}(z-z_0)+\ldots \Leftrightarrow \lim_{z\to z_0} g(z) = a_{-m}\in \C\backslash\{0\}
\end{align*}
\end{enumerate}
\end{dem}

\begin{defi}
$$f(z) = \sum_{n= -\infty}^{\infty} a_n (z-z_0)^n,\ a_n = \frac{1}{2\pi i} \int_C \frac{f(w)}{(w-z_0)^{n+1}}dw$$
Serie de Laurent centrada en $z_0$.
\end{defi}

\begin{defi}
Sea $z_0$ una singularidad de la función $f(z)$ y $\sum_{n= -\infty}^{\infty} a_n (z-z_0)^n$ su serie de Laurent centrada en $z_0$:
\begin{enumerate}
\item $f(z)$ tiene una singularidad evitable en $z_0$ si $aN=0\ \forall n<0$.
\item $f(z)$ tiene un polo de orden $m$ en $z_0$ si $m$ es el mayor natural de forma que $a_{-m}\neq 0$.
\item $f(z)$ tiene una singularidad esencial si existen infinitos $a_{-m}\neq 0$. 
\end{enumerate}
\end{defi}

\begin{lema}
Sea $z_0$ una singularidad aislada de $f(z)$ entonces $z_0$ es una singularidad evitable si y solo si $(z)$ está acotada en un entorno perforado de $z_0$.
\end{lema}

\begin{dem}
Primero demostraremos $(\Rightarrow)$.

Supongamos que $z_0$ es singularidad evitable. Entonces podemos redefinir la función para que sea analítica en un entorno de $z_0$ con lo que $f$ es acotada en un entorno perforado de $z_0$.

Demostremos ahora $(\Leftarrow)$.

Supongamos que $f(z)$ está acotada en un entorno perforado de $z_0$. Entonces $\lim_{z\to z_0} f(z)(z-z_0) =0$ y entonces $g(z) = f(z)(z-z_0)$ tiene una singularidad evitable en $z_0$.

Entonces podemos escribir $g(z) = \sum_{n=0}^\infty a_n (z-z_0)^n \Rightarrow f(z) = \frac{a_0}{a-a_0} + a_1 + a_2(z-z_0)+\cdots$

Por tanto, es o bien polo simple de $f$ o singularidad evitable de $f$ ( si $a_0=0$). Pero como $f$ está acotada $z_0$ no puede ser polo de$f$ así que concluimos que $z_0$ es singularidad evitable de $f$.$\qed$
\end{dem}


\begin{theorem}[Casorati-Weierstrass]
Si $f(z)$ tiene una singularidad esencial en $z_0$ entonces $F(D(z_0,\varepsilon\backslash\{z_0\})$ es denso en $\C$ para  todo $\varepsilon>0$, es decir, la imagen de cualquier entorno perforado de $z_0$ es denso en $\C$.
\end{theorem}

\begin{dem}
Sea $w\in\C$ y $g(z) = \frac{1}{f(z)-w}$ veamos que $g(z)$ no está acotada en cualquier entorno perforado.

Supongamos, por reducción al absurdo, que $g$ está acotada en un $D(z_0,r)\backslash\{z_0\}$ para algún $r>0$. Sabemos que $g(z)$ tiene una singularidad evitable en $z_0$ por ser acotada. Entonces podemos definir una función $g_1(z)$ tal que sea analítica en $D(z_0,r)$ con $g_1(z_0) \neq 0$.

Sea $m$ el orden del 0 de $g$ en $z_0$:
$$g(z) = (z-z_0)^m g_1(z) = \frac{1}{f(z)-w}\Rightarrow (z-z_0)^m f(z) = (z-z_0)^m w+\frac{1}{g_1(z)}$$

$$\Rightarrow \lim_{z\to z_0} (z-z_0)^m f(z) = \frac{1}{(g_1(z_0)}\Rightarrow z_0 \text{ es polo de orden m de } f\ \sharp$$

Vemos como $g(z)$ no está acotada en ningún entorno perforado de $z_0$ y en conclusión $f(z) \to w\ \forall w\in\C$.$\qed$
\end{dem}

\begin{prop}[Clasificación de singularidades]
Sea $z_0$ una singularidad aislada de $f(z)$ entonces:
\begin{enumerate}
\item $z_0$ es evitable $\displaystyle \Leftrightarrow \lim_{z\to z_0} f(z) \in\C$.
\item $z_0$ es polo de orden $\displaystyle m \Leftrightarrow \lim_{z\to z_0} f(z)(z-z_0)^m \in\C\backslash\{z_0\}$.
\item $z_0$ es esencial $\displaystyle \Leftrightarrow \not \exists \lim_{z\to z_0} |f(z)|$.
\end{enumerate}
\end{prop}

\begin{dem}
Con anterioridad hemos demostrado 1 y la implicación hacia la derecha de 2,demostremos ahora la implicación hacia la izquierda:

2.$(\Leftarrow)$ Si $\lim_{z\to z_0}|f(z)|= \infty \Rightarrow f(z)$ po está acotada en un entorno perforado de $z_0$, entonces es polo o singularidad esencial y por el Teorema de Casorati-Weierstrass $z_0$ no es singularidad esencial $\Rightarrow$ es polo.

Dado que una singularidad solo puede ser evitable, polo o esencial, demostrados 1. y 2. podemos demostrar 3. por descarte. $\qed$

\end{dem}

\begin{defi}
Diremos que $f(z)$ tiene una singularidad en $\infty$ si $f(z)$ es analítica en $\{z\in\C / |z|>r\}$ para algún $r>0$.
El tipo de singularidad de $f(z)$ en $\infty$ se define como el tipo de singularidad de $g(z) = f\left(\frac{1}{z}\right)$ en $z=0$.
\end{defi}

\begin{prop}
Sea $f(z)$ una función entera entonces:
\begin{enumerate}
\item $z_0$ es evitable $\Leftrightarrow f(z)$ es constante.
\item $z_0$ es un polo $\Leftrightarrow f(z)$ es un polinomio no constante.
\item $z_0$ es esencial $\Leftrightarrow f(z)$ no es un polinomio.
\end{enumerate}
\end{prop}

\begin{dem}
1.$(\Leftarrow)$ Si $f(z) = cte.$ entonces $g(z) = f\left(\frac{1}{z}\right)= cte.$ y $z=0$ es una singularidad evitable de $g(z)\Rightarrow z=\infty$ es singularidad evitable de $f(z)$.

$(\Rightarrow)$ Si $\infty$ es evitable en $f(z)$ entonces $z=0$ lo es de $g(z) = f\left(\frac{1}{z}\right)$ por lo que $g(z)$ es acotada en un entorno perforado de $z=0 \Rightarrow f(z)$ es acotada en $\C$ y por el teorema de Liouville $f$ es constante.

2. $\infty$ es un polo de $f\Leftrightarrow z=0$ es polo de $g(z) = f\left(\frac{1}{z}\right)$ entonces $\lim_{z\to 0}|g(z)| = \infty$. Además

$g(z) = f\left(\frac{1}{z}\right) = \sum_{n\geq 0} a_n \frac{1}{z^n}\Leftrightarrow 0$ es polo de $g \Leftrightarrow a_n=0\ \forall n\geq m,\ m\in\N\Leftrightarrow f(Z) = a_0 + a_1z + \cdots + a_mz^m$.

 Por lo que $f(z)$ es un polinomio, no constante (por 1).

3. Es consecuencia de 1 y 2. 
\end{dem}


\chapter{Teoría de residuos}

\begin{defi}
Sea $z_0$ una singularidad aislada de una función $f(z)$ y $\displaystyle \sum_{n=-\infty}^{\infty} a_n (z-z_0)^n$ su desarrollo de Laurent centrada en en $z_0$ entonces decimos que $a_{-1}$ es el residuo de $f(z)$ en $z_0$ y se denota por $\Res(f,z_0)$.
\end{defi}

\textbf{Observación:} Si $z_0$ es una singularidad evitable entonces $\Res(f,z_0) =0$. 
Si $z_0$ es una singularidad aislada de $f(z)$, $\exists r>0$ tal que $f(z)$ es analítica en $\{z\in\C / 0z|z-z_0|<r\}$ y su desarrollo de Laurent presenta unos coeficientes que vienen dados por 
$$
a_n = \frac{1}{2\pi i}\int_C \frac{f(w)}{(w-z_0)^{n+1}}dw\ \ C= C(Z_0,r)
$$
En particular $n=-1 \longrightarrow a_{-1} = \frac{1}{2\pi i}\int_C f(w)dw$ y entonces vemos como
$$\int_C f(w)dw = 2\pi i \Res(f,z_0)$$

\begin{prop}
Si $f(z)$ tiene un  polo de orden $m$ en $z_0$ entonces $\Res(f,z_0)= \frac{1}{(m-1)!}\lim_{z\to z_0}g^{(m-1)}(z)$ con $g(z) = (z-z_0)^mf(z)$.

En particular si $z_0$ es polo simple $Res(f,z_0) = \lim_{z\to z_0} (z-z_0)f(z)$
\end{prop}

\begin{dem}
Sea $z_0$ un polo de orden $m\geq 1$ de $f(z)$ entonces $\lim_{z\to z_0}(z-z_0)^m f(z) = \lim_{z\to z_0} g(z) \in \C\backslash\{0\}$.

Notemos que $z_0$ es una singularidad evitable de $g(z)$ entonces podemos escribir $g(z) = b_0 + b_1(z-z_0) + \cdots \Rightarrow$
$$f(z) = \frac{b_0}{(z-z_0)^m}+ \cdots + \frac{b_{m-1}}{z-z_0}+ b_m + b_{m+1}(z-z_0) + \cdots$$
$$\Res(f,z_0) = b_{m-1} = \frac{1}{(m-1)!}\lim_{z\to z_0} g^{(m-1)}(z)$$
Y para $m =1$ tenemos que $\Res(f,z_0) = \lim_{z\to z_0} (z-z_0)f(z)$.$\qed$
\end{dem}


\begin{prop}
Si $f(z)$ es analítica en $z_0$, no idénticamente constante y tiene un = de orden $m$ en el punto $z_0$ entonces $g(z) =  \frac{f'(z)}{f(z)}$ tiene un polo simple en $z_0$ y $\Res(g,z_0) = m$.
\end{prop}

\begin{dem}
Dado que $z_0$ es cero de orden $m$ de $f(z)$ entonces $f(z) = (z-z_0)^mf_1(z)$ con $f_1(z_0)\neq 0$ analítica.
Y entonces $f'(z)  = m(z-z_0)^{m-1}f_1(z) + (z-z_0)^mf'_1(z)$.

$$g(z) = \frac{m(z-z_0)^{m-1}f_1(z) + (z-z_0)^mf'_1(z)}{(z-z_0)^mf_1(z)} = \frac{m}{z-z_0}+\frac{f_1'(z)}{f_1(z)}$$
Y al ser $\frac{f_1'(z)}{f_1(z)}$ analítica, $z_0$ es un polo simple de g y $\Res(g,z_0) = m$.$\qed$
\end{dem}
\section{Teorema de los residuos}
\begin{theorem}[Teorema de los residuos]
Sea $f(z)$ analítica en un abierto $U$ excepto, a lo sumo, en $w_1,\ldots,w_n$ que son singularidades aisladas. Entonces para todo $\gamma$ camino cerrado tal que $n(\gamma,z_0) \neq 0$ $\forall z\notin U$ y con $w_j \notin \gamma^*, j= 1,\ldots,n$ se tiene que:
$$\int_\gamma f(z)dz = 2\pi i \sum_{j\geq 1} n(\gamma,w_j) \Res(f,w_j)$$
\end{theorem}

\begin{dem}[por hacer]
\end{dem}

\begin{lema}[Lema de Jordan]
Sea $f$ analítica en el semicírculo $\Gamma_R = \{z/ Re^{i\theta},0\leq \theta\leq \pi\}$ y sea $M(R) = \max\{|f(z)|/ z\in \Gamma_R\}$. Si $a>0 \Rightarrow$
$$\abs{\int_\Gamma f(z)e^{iaz}dz} \leq \frac{\pi M(R)}{a}(1-e^{-aR})$$
\end{lema}

\begin{dem}
\begin{align*}
&\abs{\int_\Gamma f(z)e^{iaz}dz} = \abs{\int_0^\pi f(Re^{i\theta})e^{iaRe^{i\theta}}Rie^{i\theta}d\theta} \int_0^\pi \underset{*}{\leq} \int_0^\pi \underset{\leq M(R)}{\abs{f(Re^{i\theta})}}e^{-aR\sin\theta}Rd\theta\\
&\leq RM(R)\int_0^\pi e^{-aR\sin\theta}d\theta = 2RM(R)\int_0^{\pi/2} e^{-aR\sin\theta}d\theta\\
&\underset{**}{\leq} 2RM(R)\int_0^{\pi/2} e^{-aR\frac{2}{\pi}\theta}d\theta = \frac{\pi M(R)}{a}(1-e^{-aR})
\end{align*}
$$
* \abs{e^{iaRe^{i\theta}}} = \abs{e^{iaR(\cos\theta + i\sin\theta)}} = e^{-aR\sin\theta}\abs{e^{iaR\cos\theta}}=e^{-aR\sin\theta}
$$
$**$ Si llamamos $g(\theta) = \sin\theta -\frac{2}{\pi}\theta$ con $\theta \in[0,\pi/2]$

Vemos como $g(0) =g(\pi/2)=0$, $g'(\theta) = \cos\theta -\frac{2}{\pi}$, $g''(\theta) = -\sin\theta \leq 0$.

Entonces $g(\theta) \geq 0\ \forall \theta \in[0,\pi/2] \Leftrightarrow \sin\theta \geq \frac{2}{\pi}\theta$. $\qed$
\end{dem}

\begin{prop}
Sea $f(z)$ una función analítica en $\{z/|z|>r,\Im(z)\geq 0\}\forall r>0$ y consideremos el arco semicircular de centro $0$ y radio $R$ positivamente orientado dado por
$\gamma_R = \{z\in\C/ z = Re^{i\theta},0\leq \theta\leq \pi\}$.
[falta completar]
\end{prop}
\section{Consecuencias del Teorema de los Residuos}
\begin{prop}[Principio del argumento]
Sea $f(z)$ una función analítica en un abierto $U$ y no idénticamente nula en ninguna componente conexa de $U$, si $\gamma$ es el camino cerrado en $U\backslash Z(f)$ tal que $n(\gamma,z) = 0\ \forall z \notin U\Rightarrow$
$$n(f\circ\gamma,0) = \sum_{w\in Z(f)} m(f,w)n(\gamma,w)$$
Con $m(f,w)$ el orden del cero de $w$.
\end{prop}

\begin{dem}
$$n(\gamma,z_0) = \frac{1}{2\pi i}\int_\gamma \frac{dz}{z-z_0}$$
La función $\frac{f'(z)}{f(z)}$ analítica en $U$ excepto en $Z(f)$. Así por el teorema de los residuos, obtenemos que 
$$\int_\gamma \frac{f'(z)}{f(z)} dz = 2\pi i \sum_{w\in Z(f)}n(\gamma,w)\Res(\frac{f'(z)}{f(z)},w) = 2\pi i\sum_{w\in Z(f)}n(\gamma,w)m(f,w)$$

Por otra parte  sea $\gamma:[a,b]\to \C$, por definición de índice
\begin{align*}
n(f\circ \gamma,0) = \frac{1}{2\pi i}\int_{f\circ\gamma}\frac{dz}{z}&= \frac{1}{2\pi i}\int_a^b \frac{1}{f(\gamma(t))}f'(\gamma(t))\gamma'(t)dt \\
&= \frac{1}{2\pi i}\int_\gamma \frac{f'(z)}{f(z)}dz\ \qed
\end{align*}
\end{dem}

>
\begin{theorem}[Teorema de Rouché]
Sean $f(Z)$ y$g(z)$ dos funciones analíticas en un abierto $U$ y no idénticamente nulas en ninguna componente conexa de$U$, consideramos $\gamma$ un camino cerrado en $U$ tal que $n(\gamma,z)=0$ $\forall z\notin U$.

\begin{align*}
&\text{Si }\abs{f(z)-g(z)} < \abs{f(z)}\ \forall z\in\gamma^*\Rightarrow  \\
&\sum_{w\in Z(f)}m(f,w)n(\gamma,w) = \sum_{w\in Z(g)}m(g,w)n(\gamma,w)
\end{align*}
\end{theorem}

\begin{dem}
Por la desigualdad supuesta observamos en primer lugar que $\gamma$ no pasa por ningún cero de $Ff$. Así podemos definir $h(z) = 1 +\frac{g(z)-f(z)}{f(z)}$ en un cierto entorno de $\gamma^*$.

Además si $z\in\gamma^*$ entonces $\abs{h(z)-1} =\frac{\abs{g(z)-f(z)}}{\abs{f(z)}}<1\ \forall z\in\gamma^*$.

Es decir, la curva $h\circ\gamma$está incluida en $D(1,1)$ y, por tanto, no le da vueltas al origen, $n(h\circ\gamma,0)=0$.

Por otra parte $h(z) f(z) = g(z)$ por lo tanto
\begin{align*}
&n(g\circ\gamma,0) = n((h\cdot f)\circ \gamma,0) =\frac{1}{2\pi i}\int_\gamma\frac{(h(z)f(z))'}{h(z)f(z)}dz =\\
&=\frac{1}{2\pi i}\left(\int_\gamma\frac{h'(z)}{h(z)}dz +\int_\gamma \frac{f'(z)}{f(z)dz} \right)= \cancelto{0}{n(h\circ\gamma,0)} + n(f\circ\gamma,0) = n(f\circ\gamma,0)  
\end{align*}
Entonces por el principio del argumento podemos concluir que $$\sum_{w\in Z(f)}m(f,w)n(\gamma,w) = \sum_{w\in Z(g)}m(g,w)n(\gamma,w).$$ $\qed$
\end{dem}


\begin{theorem}[Teorema de Hurwitz]
Supongamos que $\{f_n\}$ es una sucesión de funciones analíticas es un abierto $U$ que converge uniformemente sobre los compactos de $U$ a una función $f(z)$.

Consideremos $\overline{D(z_0,r)}\subset U$ y asumimos que $f(z)\neq 0$ para todo $z$ tal que $\abs{z-z_0}=r$, entonces
$$
\exists N\in\N / f_n(z) \text{ tiene el mísmo número de ceros que } f(z) \text{ en } D(z_0,r)\ \forall n\geq N
$$
\end{theorem}

\begin{dem}
Observemos que $f$ es analítica en $U$.

Consideremos $\varepsilon = \min\{\abs{f(z)}/\abs{z-z_0}= r\}>0$.

Por la convergencia uniforme, $\exists N \in N$ tal que $\abs{f_n(z)-f(z)}<\varepsilon \leq \abs{f(z)},\ \forall z\in C(z_0,r)$.

Por el teorema de Rouché, tomando $g(z) = f_n(z)$ vemos como $f_n$ y $f$ tienen el mismo número de ceros en $D(z_0,r)$.$\qed$
\end{dem}

\begin{prop}
Sea $f(z)$ una función analítica no constante en cada componente conexa de un abierto $U$ y con un cero de orden $m$ en $z_0$ entonces 
\begin{enumerate}
\item $\exists \varepsilon>0$ tal que $\overline{D(z_0,\varepsilon)}\subset U$ tal que $f(z)$ y $f'(z)$ no se anulan en $D(z_0,\varepsilon)\backslash\{z_0\}$.
\item Si $p = \min\{\abs{f(z)}/ \abs{z-z_0}=r\}$ con $\varepsilon >0$ verificando el apartado anterior y con $w$ tal que $0<\abs{w}<p$ entonces existen $m$ puntos distintos en $D(z_0,\varepsilon)$ de imagen $w$.
\item Existe un abierto $V$ con $z_0\in V$ tal que si $z\in V\backslash\{z_0\}\subset U$ entonces $f(z)\neq 0$ y existe exactamente $m$ puntos $w_k\in V\backslash\{z_0\}$ tal que $f(w_k) = f(z),\ k = 1,2,\ldots,m$.
\end{enumerate}
\end{prop}

\begin{dem}
\textbf{1.} Por reducción al absurdo supongamos que no existe $\varepsilon>0$ tal que se cumplan las condiciones.

Entonces $z_0$ es punto de acumulación de ceros de $f(z)$ o $f'(z)$ y por el principio de identidad $f(z)=0$ o $f`'(z)=0$ con lo que $f$ es constante.$\sharp$

\textbf{2.} Sea $p = \min\{\abs{f(z)}/ \abs{z-z_0}=r\}$, $w\in \C / 0<\abs{w}<p$ y sea $g(z) = f/z)-w$

Entonces $\abs{w} <p\leq \abs{f(z)}\ \forall z\in \gamma = C(z_0,\varepsilon)$, $\abs{w} = \abs{f(z)-g(z)}$.

Por el teorema de Rouché, $f$ y $g$ tienen el mismo número de ceros en $D(z_0,\varepsilon)$ si $f(z) = w$ tiene exactamente $m$ ceros.

Además todos estos ceros son distintos ya que en caso contrario $f'=0$.

\textbf{3.} Tomemos $\varepsilon$ y $p$ de los apartados anteriores.  Entonces , sea $V = D(z_0,\varepsilon)\cap f^{-1}(D(0,p))\ni z_0$ y $V\subset  D(z_0,\varepsilon)$.

Por la elección de $varepsilon$, $f(z) \neq 0\ \forall z\in V\backslash\{z_0\}$.

Además, si $z\in V\backslash\{z_0\}$ entonces se tiene que $0<\abs{f(z)}<p$ y por el apartado 2. $w=f(z)$ y existen $m$ puntos $w_0,\ldots,w_m$ en $D(z_0,\varepsilon)$ de imagen $w$ con $0<\abs{w}<p$ y entonces, por definición de $V$, los puntos $w_k$ están en $V$ para $k = 1,2,\ldots,m$.$\qed$
\end{dem}


\begin{theorem}[Teorema de la aplicación abierta]
Sea $f(z)$ una función analítica en un abierto $U$ y no idénticamente constante en ninguna componente conexa de $V$. Si $V$ es un abierto de $U$ entonces $f(V)$ es un abierto de $\C$.
\end{theorem}

\begin{dem}
Sea $V$ un abierto de $U$ y $z_0\in V$ veamos que $f(z_0)$ es interior a $f(V)$. 

Consideremos $g(z) = f(z) -z_0$ analítica en $U$ y no idénticamente constante en ninguna componente conexa y con $g(z_0) = 0$, y sea $m\geq 1$  el orden del cero de g$g$.

Tomemos $\varepsilon >0$ y por la proposición anterior, con $D(z_0,\varepsilon )\subset V$.

Entonces $\forall w \in D(0,p)\backslash \{z_0\} \exists m$ puntos  $w_1,w_2\ldots,w_m$ en $D(0,p)$ tal que $g(w_k)=w$.

Por tanto $D(0,p)\subset g(V)$ o equivalentemente $D(f(z_0),p) \subset f(V)$ con lo que $f(z_0)$ es interior a $f(V)$ y $f(V)$ es abierto.$\qed$
\end{dem}


\begin{prop}
Sea $f(z)$ una función analítica en $z_0$
\begin{itemize}
\item Si $f`(z_0) \neq 0$ entonces existe un entorno donde $f(z)$ es inyectiva.
\item Si $f'(z_0)=0$ entonces $f(z)$ no es inyectiva en ningún entorno de $z_0$.
\end{itemize}
\end{prop}

\begin{dem}
\textbf{1.} Supongamos $f'(z_0)\neq 0$ y consideremos $g(z) = f(z)-f(z_0)$ analítica en $z_0$ con  $g(z_0) = 0$ y $g'(z_0) \neq 0$.

Así por el apartado 3 de la proposición anterior existe un abierto $V\ni z:_0$ tal que $\forall z\in V\backslash\{z_0\}$ existe un único$w \in V\backslash\{z_0\}$ de imagen $g(z)$ entonces $f$ es inyectiva en $V$.

\textbf{2.} Supongamos que $f'(z_0)=0$ entonces $g(z) = f(z)-f(z_0)$ analítica en $z_0$ y $g(z_0)= g'(z_0) = 0$.Es decir $z_0$ es un cero de orden mayor que 2. Y por el apartado anterior $g$ no es inyectiva y por lo tanto $f$ tampoco.$\qed$

\end{dem}


\begin{theorem}
Sea $f(z)$ una función analítica e inyectiva en un abierto $U$ entonces $f^{-1}(z)$ es analítica en el abierto $f(U)$.
\end{theorem}

\begin{dem}
Por el teorema de la función abierta, $f(U)$ es un conjunto abierto si $U$ es abierto, por tanto $f^{-1}: f(U)\longleftrightarrow U$ está bien definida en un conjunto abierto.

Veamos que $f^{-1}$ es analítica en $f(U)$. Veamos que $f^{-1}$ es derivable en todo punto de $f(U)$.

Sea $w_0\in f(U)$:
$$
\lim_{w\to w_0}\frac{f^{-1}(w) - f^{-1}(w_0)}{w-w_0} = lim_{z\to z_0)} \frac{z-z_0}{f(z)-f(z_0)} *= \frac{1}{f'(z_0)}
$$

$$\left[\begin{array}{ll}
f(z) = w &\Rightarrow f^{-1}(w) = z\\
 f(z_0) = w_0 &\Rightarrow f^{-1}(w_0) = z_0\\
 f^{-1}(z) &= \frac{1}{f'(f^{-1}(z))}
\end{array}\right]$$
\end{dem}

\chapter{Funciones enteras}
\begin{defi}[Convergencia de productos infinitos]
Sea $z_1,z_2,\ldots z_n$ una sucesión de números complejos no nulos. Consideremos $p_n =\prod_{k=1}^n z_k$. Si existe $\\displaystyle lim_{n\to \infty}p_n = p$ con $p\neq 0$ diremos que el producto infinito $\prod_{n\geq 1}z_n$ converge  a $p$. En caso contrario diremos que el producto no converge.

Si un producto infinito tiene una cantidad finita de términos iguales a $0$ y el resto verifica la definición anterior, también diremos que $\prod_{n\geq 1}z_n$ converge. De hecho , si existe $m\in \N$ tal que $\prod_{n\geq m+1}z_n$ verifica la definición inicial, se dice que el producto $\prod_{n\geq 1}z_n$ es convergente a $p = z_1z_2\cdots z_m\prod_{n\geq m+1}z_n$.
\end{defi}

\begin{defi}
LLamaremos factores canónicos o elementales a las funciones 
$$ E_0(z) = 1-z$$
y
$$E_m(z) = (1-z)\exp\left(z+\frac{z^2}{2}+\cdots + \frac{z^m}{m}\right),\ m\in\N$$
\end{defi}

\textbf{Propiedades}
\begin{itemize}
\item $E_m(z)$ es una función entera para cada $m = 0,1,2,\ldots$
\item $E_m(z)$ tiene un único cero simple en $z=1$ para cada   $m = 0,1,2,\ldots$
\item si $\abs{z}$ entonces $\displaystyle \lim_{m \to \infty} E_m(z) =(1-z) exp(-log(1-z))=1$.
\item Para todo $m\in \N \cup \{0\}$, se cumple que
$$\abs{1-E_m(z)} \leq \abs{z}^{m_1} \text{ para } z \text{ tal que } \abs{z}\leq 1.$$
\end{itemize}


\begin{theorem}
Sea $\{a_n\}_{n\geq 1}$ una sucesión de números complejos no nulos tales que $\displaystyle\lim_{n\to \infty} \abs{a_n} = \infty$. Si $m_n \geq 1$, el producto $f(z) = \prod_{n\geq1}E_{m_n}\left(\frac{z}{a_n}\right)$ es una función entera cuyos ceros vienen dados por $a_1,a_2,\ldots$.
\end{theorem}

\begin{defi}
Sea $\{a_n\}_{n\geq 1}$ una sucesión de números complejos no nulos tales que $\displaystyle\lim_{n\to \infty} \abs{a_n} = \infty$. El coeficiente
$$\mu = \inf\{k>0 / \sum_{n\geq 1}\frac{1}{\abs{a_n}^k} < \infty\}$$
se conoce como \textit{exponente de convergencia} de la sucesión $\{a_n\}_{n\geq 1}$.
\end{defi}

\begin{theorem}
Sea $\{a_n\}_{n\geq 1}$ una sucesión de números complejos no nulos tales que $\displaystyle\lim_{n\to \infty} \abs{a_n} = \infty$ y su exponente de convergencia sea $\mu < \infty$. Si $h$ es un entero no negativo tal que $h > \mu-1$, entonces el producto infinito $\displaystyle \prod_{n=1}^\infty E_h\left(\frac{z}{a_n}\right)$ define una función entera cuyos ceros vienen dados por $a_1,a_2,\ldots$.

Cuando $\mu$ es entero y $\sum_{n\geq 1}\frac{1}{\abs{a_n}^\mu} < \infty$ podemos tomar $h = \mu -1$.
\end{theorem}


\begin{theorem}[Factorización de Weierstrass]
Sea $f(z)$ una función entera no idénticamente nula con un cero en $z=0$ de orden $m\in N \cup \{0\}$( tomamos m=0 si $f(0) \neq 0$) y otros ceros $a_1,a_2,\ldots$ repetidos según su multiplicidad. Entonces 
$$f(z) = e^{g(z)}z^m \prod_{n\geq 1} E_{m_n}\left(\frac{z}{a_n}\right)$$
para una cierta función entera $g(z)$ y enteros $m_n \geq n-1$.

En general, si el exponente de convergencia es finito podemos tomar $m_n = h> \mu -1$.
\end{theorem}

\begin{defi}
Diremos que una función $f(z)$ es meromorfa en un abierto $U$ cuando $F(z)$ es analítica en $U$ excepto posiblemente en singularidades aisladas que son polos.
\end{defi}

\begin{col}
Una función $f(z)$ es meromorfa en $\C$ si, y silo si, $f(z) = \frac{g(z)}{h(z)}$ con $g(z)$ y $h(z)$ funciones enteras y siendo $h(z)$ no idénticamente nula.
\end{col}

\begin{dem}
Si $f(z) = \frac{g(z)}{h(z)}$ con $g(z)$ y $h(z)$ funciones enteras y siendo $h(z)$ no idénticamente nula, entonces $f(z)$ es meromorfa donde los polos de $f(z)$ son los ceros de la función $h(z)$.

Recíprocamente, si $f(z)$ es meromorfa, denotemos por $b_1,b_2,\ldots,b_n,\ldots$ a sus polos. Utilizando el teorema de factorización de Weierstrass, construimos una función entera $h(Z)$ cuyos ceros sean justamente los polos $b_n$,. En tal caso, la función $g(z):=f(z)h(z)$ presenta singularidades evitables en los puntos $b_n$ y al eliminarlas se consigue una función entera. Luego $f(z) = \frac{g(z)}{h(z)}$. $\qed$
\end{dem}

\begin{defi}
dada $f(z)$ una función entera, denotaremos orden ( u orden de crecimiento) de $f(z)$ al valor $\lambda:=\lambda_1 = \lambda_2$ donde
$$
\lambda_1 = \limsup_{r\to \infty}\frac{\ln(\ln M_f(r))}{\ln r}
$$
y
$$
\lambda_2 = \inf\{a\geq 0 /\ \exists r >0 \text{ tal que } \abs{f(z)}\leq e^{\abs{z}^a}\ \forall z\in\C \backslash B(0,r)\}
$$
\end{defi}

\begin{dem} Veamos que efectivamente $\lambda_1 = \lambda_2$.

Primero demostremos que $\lambda_1 \geq \lambda_2$:
Si $\lambda_1 = \infty$ es trivial por lo que supondremos $\lambda_1<\infty$.

Por definición de $\lambda_1$, dado $\varepsilon >0\ \exists r_0>0\ /\  \frac{\ln\ln M_f(r)}{\ln r} \leq \lambda_1 + \varepsilon\ \forall r\geq r_0$.

\begin{align*}
& \Rightarrow \ln\ln M_f(r) \leq (\lambda_1+\varepsilon)\ln r \Rightarrow \ln\ln M_f(r) \leq \ln r^{(\lambda_1+\varepsilon)} \\
& \Rightarrow  \ln M_f(r) \leq r^{(\lambda_1+\varepsilon)}\Rightarrow M_f(r) \leq e^{r^{\lambda_1+\varepsilon}}\\
& \text{si } \abs{z} = r \Rightarrow \abs{f(z)} \leq e^{\abs{z}^{\lambda_1 + \varepsilon}}
\end{align*}
Y como podemos escoger $\varepsilon$ de forma arbitraria, $\lambda_1 \geq \lambda_2$.

Primero demostremos que $\lambda_2 \geq \lambda_1$:
Si $\lambda_2 = \infty$ es trivial por lo que supondremos $\lambda_2<\infty$.

Por la definición de $\lambda_2$ dado $\varepsilon>0$, $\exists r>0$ suficientemente grande tal que $\abs{f(z)}\geq e^{\abs{z}^{\lambda_2+\varepsilon}}$ $\forall z\in \C\backslash D(0,r)$.

\begin{align*}
&\Rightarrow \ln\abs{f(z)} \leq \abs{z}^{\lambda_2+\varepsilon} \Rightarrow \ln\ln \abs{f(z)} \leq (\lambda_2+\varepsilon)\ln\abs{z}\\
&\text{si }\abs{z}=r \Rightarrow \ln\ln M_f(r) \leq (\lambda_2+\varepsilon)\ln r \Rightarrow \frac{\ln\ln M_f(r)}{\ln r} \leq \lambda_2 + \varepsilon
\end{align*}
Y como podemos escoger $\varepsilon$ de forma arbitraria, $\lambda_2 \geq \lambda_1$.

Con lo que concluimos que $\lambda_1 =\lambda_2$. $\qed$

\end{dem}
\textbf{Nota}

Sea $f(z)= \sum_{n\geq 0} a_n z^n$ una función entera no constante, entonces el orden de $f(z)$ viene también dado por 
$$\lambda = \limsup_{n\to \infty} \frac{\ln n}{-\ln \abs{a_n}^{1/n}} = \limsup_{n\to \infty} \frac{n\ln n}{\ln \left(\frac{1}{\abs{a_n}}\right)}
$$

\textbf{Ejemplos}

\begin{itemize}
\item El orden de un polinomio es $0$.
\item El orden de $e^{a_0+a_1z+\cdots+a_nz^n}$ con $a_n \neq 0$ es $n$.
\item El orden de $e^{e^z}$ es $\infty$.
\item El orden de $\sin z, \cos z, \sinh z, \cosh z$ es $1$.
\end{itemize}

\begin{prop}
Sea $f(z)$ una función entera no constante de orden finito $\lambda$. Sean $a_1,a_2,\ldots$ los ceros de $f(z)$ repetidos según su multiplicidad. Entonces $\mu \leq \lambda$, donde $\mu$ es el exponente de convergencia asociado a esta sucesión.
\end{prop}


\begin{theorem}[Factorización de Hadamard]
Sea $f(z)$ una función entera de orden finito $\lambda$ con un cero en $z=0$ de orden $m \in \N \cup\{0\}$ y otros ceros $a_1,a_2,\ldots$ repetidos según su multiplicidad. Entonces
$$f(z) = e^{P(z)}z^m \prod_{n\geq 1} E_{m_n}\left(\frac{z}{a_n}\right)$$
donde $P(z)$ es un polinomio de grado menor o igual que $\lambda$ y 
$$
h  = \left\lbrace\begin{array}{cl}
	\mu - 1 & \text{ si } \mu \in \Z, \sum_{n\geq 1}\frac{1}{\abs{a_n}^\mu} < \infty\\
	\mu & \text{ si } \mu \in \Z, \sum_{n\geq 1}\frac{1}{\abs{a_n}^\mu} = \infty\\
	\left[\mu\right]  & \text{ si } \mu \notin \Z
\end{array}\right.
$$
Siendo $\mu$ el exponente de convergencia asociado a $\{a_n\}$.
\end{theorem}
\end{document}
