\documentclass{book}


\usepackage{amsmath, amsfonts, amssymb, amsthm} %paquetes matematicos
\usepackage{mathtools}
\usepackage{cancel}

%paquetes idioma
\usepackage[T1]{fontenc}
\usepackage[utf8]{inputenc}
\usepackage[spanish]{babel}
\usepackage{csquotes}

\usepackage{graphicx} %añadir imagenes

\usepackage{listings} %codigo python resaltado
\usepackage{xcolor} %colores

%\usepackage[a4paper,left=3cm,right=2.5cm,top=2.5cm,bottom=2.5cm]{geometry} %margenes y geometria de la hoja

\usepackage{verbatim} %entorno comment
\usepackage{subfig} %subfiguras
\usepackage{float} % parametro H en figuras

\usepackage[hidelinks]{hyperref} %Urls

\usepackage[affil-it]{authblk}

\usepackage{tikz}
\usepackage{tkz-euclide}
\usetikzlibrary{positioning,decorations.markings}

\tikzset{
every picture/.append style={
  execute at begin picture={\deactivatequoting},
  execute at end picture={\activatequoting}
  }
}

%paquetes de tablas
\usepackage{tabularx}
\usepackage{multirow}

\newcolumntype{M}{>{$}c<{$}} %Tipo de columna matematica

%Lineas viudas y huerfanas
\clubpenalty=10000
\widowpenalty=10000

\pretolerance=10000
\tolerance=10000

%Datos portada

\author{-----}
\affil
{
	Universidad de Alicante
    \vskip -1cm
}
% \author{Óscar Riquelme Moya}
% \affil
% {
%     Departamento de física aplicada\\
%     Universidad de Alicante
%     \vskip -1cm
% }

%Fuentes y colores para el highlight
\DeclareFixedFont{\ttb}{T1}{txtt}{bx}{n}{9} %bold
\DeclareFixedFont{\ttm}{T1}{txtt}{m}{n}{9}  %normal

\definecolor{font}{rgb}{0.2,0.1,0.2}
\definecolor{background}{rgb}{0.95,0.95,1}
\definecolor{deepblue}{rgb}{0,0,0.5}
\definecolor{deepred}{rgb}{0.6,0,0}
\definecolor{deepgreen}{rgb}{0,0.5,0}
\definecolor{grey}{rgb}{0.3,0.3,0.3}
\definecolor{red}{rgb}{1,0,0}

%estilo para el codigo python
\lstdefinestyle{Python}{language=python,
		backgroundcolor=\color{background},
		basicstyle=\ttm\color{font},             
		keywordstyle=\ttb\color{deepblue},  
		stringstyle=\color{deepgreen},
		showspaces=false,
		showstringspaces=false,
		tabsize=4,
		numbers=left,
		otherkeywords={self},
		emph = {__init__, as},
		emphstyle=\ttb\color{deepred},
		numberstyle=\color{grey},
		numbersep=6pt,
		stepnumber=1,
		rulecolor=\color{black},
		frame=single,
		framexleftmargin=20pt}
		


%Cambiamos el nombre de las tablas de Cuadro a Tabla
\AtBeginDocument{%
  \renewcommand\tablename{Tabla}
}

\newcommand{\parcial}[2]{
\frac{\partial #1}{\partial #2}
}


\newcommand{\parcialn}[3]{
\frac{\partial ^#3 #1}{\partial #2^#3}
}



%comando para hacer figuras rapidamente
% \fig[escala]{raiz/directorio../archivo}{caption}{label}
\newcommand{\fig}[4][1]{
\begin{figure}[H]
\begin{center}
\includegraphics[scale=#1]{#2}
\caption{#3}\label{#4}
\end{center}
\end{figure}
}

%Comandos para referenciar rápicdamente figuras tablas y ecuaciones
\newcommand{\rfig}[1]{
(figura \ref{#1})
}

\newcommand{\rtabla}[1]{
(tabla \ref{#1})
}
\newcommand{\req}[1]{
(ecuación \ref{#1})
}
\newcommand{\abs}[1]{\left| #1\right|}

%Entorno paar escribir codigo python
\lstnewenvironment{Python}{
\lstset{style=Python}}{}

\newcommand{\Pythonline}[1]{
\lstinline[style=Python]|#1|}



\setlength{\parskip}{0.5em}
\usepackage{titlesec}
\titleformat{\chapter}[block]
  {\normalfont\Huge\bfseries}{Bloque \thechapter \\ }{1em}{\huge}

%Teoremas matemáticos
\newtheorem{defi}{Definición}[chapter]
\newtheorem{theorem}{Teorema}[chapter]
\newtheorem{prop}{Proposición}[chapter]
\newtheorem*{dem}{Demostración}
\newtheorem{col}{Colorario}[chapter]
\newtheorem{lema}{Lema}[chapter]
\newtheorem{obs}{Observación}[chapter]
\newtheorem{ejemplo}{Ejemplo}[chapter]
\captionsetup[subfigure]{labelformat=empty}

\newcommand{\R}{\mathbb{R}}
\newcommand{\C}{\mathbb{C}}
\newcommand{\N}{\mathbb{N}}
\newcommand{\Z}{\mathbb{Z}}


\title{Análisis de variable compleja}

\begin{document}
\maketitle
\tableofcontents
\begin{comment}
\chapter*{Preámbulo}
\addcontentsline{toc}{chapter}{Preámbulo}
Estos apuntes están destinados a complementar los apuntes tomados por
los estudiantes de la asignatura homónima perteneciente al segundo año del
Grado en Física de la Universidad de Alicante.
Es un texto realizado fundamentalmente a partir de las notas tomadas
durante las lecciones impartidas por el profesor Tijani Pakhrou, del departamento
de Análisis Matemático, durante el curso académico 2021-2022, destinado exclusivamente a estudiantes y sin ánimo de lucro.
No está exento de erratas. La edición de estos apuntes se remite a la fecha de compilación que aparece en la portada. El último tema es el que menos horas hemos echado por falta de las mismas. 

Para obtener una copia del código fuente o para comunicar posibles erratas, o colaborar de cualquier forma para mejorar estos apuntes utilizad el siguiente repositorio de Github: {\color{blue}\url{https://github.com/Oscar015/Analisis-funcional}}

\end{comment}

\chapter{Introducción de topología} 

\begin{defi}
Sea $X$ un conjunto no vacío, una familia $\tau$ de subconjuntos de $X$ ( i.e. $\tau \subset \mathcal{P}(X)$) es una \textbf{topología} sobre $X$ si se cumplen las condiciones siguientes:

\begin{description}
\item[T1)] $\emptyset \in\tau$ y $X in \tau$
\item[T2)] Si $A_1$ y $A_2$ son dos conjuntos arbitrarios de $\tau$ entonces
$$ A_1 \cap A_2 \in\tau$$
\item[T3)] Si $\mathcal{S} = \{A_i\}_{i\in I}$ es una subfamilia cualquiera de $\tau$ entonces
$$ \bigcup_{i\in I}A_i \in \tau.$$
\end{description}
\end{defi}

El par $(X,\tau)$ se llama espacio topológico. Los conjuntos $A \in \tau$ se llaman abiertos de la topología $\tau$ , y los elementos de $X$ son los puntos del espacio.

Cuando no hay posible confusión acerca de la topología en $X$ a la que nos estamos refiriendo, designaremos el espacio $(X,\tau)$ solo por
$X$.

\begin{obs} $i\in I$ denota un indice arbitrario que puede ser numerable por ejemplo $i \in \N$ o no numerable, $i \in \R$.\end{obs}

\begin{obs} 
En el mismo conjunto $X$ podemos definir distintas topologías $\tau_1$, $\tau_2$, dando lugar a distintos espacios topológicos $(X,\tau_1)$,$(X,\tau_2)$.
\end{obs}

\end{document}